\documentclass[lang=cn,11pt]{elegantbook}
\usepackage[utf8]{inputenc}
\usepackage[UTF8]{ctex}
\usepackage{amsmath}%
\usepackage{amssymb}%
\usepackage{esint}%
\usepackage{mathrsfs}


\title{Math 395: Honor Analysis I}
\subtitle{Covering Anlysis on Manifold 2-5, instructed by Alex Wright}
\author{Qiulin Fan}
\date{2024 Fall}

\extrainfo{Notice that the note contain chunks of Chinese languages. I feel sorry if you could not read it cuz of that.}

\logo{assets/M.jpg}
\cover{assets/M.jpg}

% modify the color in the middle of titlepage
\definecolor{customcolor}{RGB}{32,178,170}
\colorlet{coverlinecolor}{customcolor}

\begin{document}

\maketitle
\frontmatter
\tableofcontents

\mainmatter

\chapter{Review of topology in $\mathbb{R}^n$}


\section{Compactness in $\mathbb{R}^n$}
\subsection{Finite Intersecion property and Heine-Borel}
\begin{theorem}
    \textbf{cpt 集的任意闭子集都继承 cpt 性.}
\end{theorem}

\begin{theorem}{Finite Intersecion property} \label{Finite Intersecion property}
   任意一堆(even unctbl个) cpt 集,只要知道它们中\textbf{任意有限个的交集都不是空的,那么它们整体的交集也不是空的. }
\end{theorem}


\begin{theorem}
    \textbf{Metric space 中,cpt 和 seq cpt 是等价的.}
    (但是一般的 topological space 中,并不等价且无蕴含关系)
    review: sequential compact 就是这个集合中的任意序列一定存在收敛子序列,并且收敛至这个集合中的一点.
\end{theorem}

\begin{theorem}{Nested Box property} \label{Nested Box property}
    这是 $\mathbb{R}^n$ 的一个性质。意思是 $\mathbb{R}^n$ 中的任意 nested 的闭区间的序列,$\cap_{i=1} ^\infty [a_i, b_i]$ 一定是非空的. (可能是一点,也可能是一个闭区间.) 
\end{theorem}

\begin{theorem}{Heine-Borel} \label{Heine-Borel}
    textbf{$\mathbb{R}^n$ 中,closed + bdd 等价于 cpt (等价于 seq cpt).} \\
(但是这个性质非常局限,因为它甚至不能推广至 metric space. 它在普遍的 metric space 中并不成立.)
\end{theorem}

一个重要例子:
\subsection{$l^{\infty}(\mathbb{N})$}

\begin{definition}{$l^{\infty}(\mathbb{N})$} \label{linfinity}
    所有 bouded sequence in $\mathbb{R}$ 构成的空间,其中 $\mathbb{N}$ 的意思是 index on自然数集,也可以取其他 countable 的集合来作为 index: $l^{\infty}(X)$ .
\end{definition}


这个空间采取\textbf{ $\sup$ norm metric:$d((x_n), (y_n)) = \sup_{n} |x_n - y_n| $. 也就是这两个 sequence 逐项差的 sup.}

这个 metric space 中的单位球是 closed 且 bounded 的(显然).  

BTW: 这个 $l^{\infty}(\mathbb{N})$ 空间也是一个 infinite dimension 的 vector space,我们在 LADR 中已经学过. 并且,我们可以给他赋范,也就是 sup norm 的由来:\textbf{$||x_n||_{\infty} = sup_{n}|x_n|$},而它的 metric 其实是它的 sup norm induce 出来的. 这个 sup norm 其实就是这个序列和零序列之间的 sup norm metric. 

\textbf{并且,在 sup norm 下作为一个 normed vector space,它是一个 Banach space。
(hw1: 任意的 normed vector space 在 $d =  ||x-y||$ 下都是一个 metric space. 如果这个 norm 使得它是 complete metric space,那么就称之为一个 Banach space.)}

然而,这个单位球并不是 cpt 的。我们知道,metric space 中 seq cpt 等价于 cpt. 所以我们可以构造一个序列 in $B_1(0)$: 考虑 $((1,0,0,\cdots), (0,1,0,\cdots),(0,0,1,\cdots),\cdots)$ ,其中每个序列的第 $n$ 项是 1 其他都是 0. 很显然这个序列并没有收敛子序列,因为其中每一项和其他之间 $d_{\infty}$ 都是1. 所以它并不 cpt.


\section{Totally-Compactness and Completeness}
\subsection{cpt. = complte + ttlbdd.}
我们知道在 $\mathbb{R}^n$ 里面,cpt = seq cpt = closed + bounded.
但是\textbf{在一般的 metric space 里面,虽然还是有 cpt = seq cpt,却不等价于有界闭了。}

不过我们仍然希望把 cptness 这个 topology 的概念转换成更加直观的概念以服务于 analysis 的目的。Fortunately 这是可行的:我们现在定义

\begin{definition}{Totally Bounded} \label{ttlbdd}
    任取 $\epsilon > 0$,我们都可以用 finitely many 个 $\epsilon$-ball 来 cover 这个集合,则称它是 Totally Bounded 的.
\end{definition}

\begin{definition}{Complete (或称 Cauchy-complete)} \label{cplt}
    即这个集合中的任意 Cauchy sequence 都收敛(至这个集合里的某个点)
    \\Rmk: 我们知道 \textbf{metric space 的任意子集也是 metric space},by induced metric.
\end{definition}

\begin{remark}
complete 即这个集合里任意 Cauchy 序列都收敛于它自己的某个点,ttl bounded 即对于任意半径都可以用有限个这样半径的球来覆盖这个集合.
\end{remark}



现在我们证明这个重要定理:
\begin{theorem}
    \textbf{任意 metric space 中,cpt = seq cpt = complete + ttl bounded}
\end{theorem}

\subsection{Lebesgue Covering Thm}
为了证明这个东西我们需要一个 Lemma,其实这个 Lemma 叫做 \textbf{Lebesgue Covering Thm}:  
\begin{lemma}{Lebesgue Covering Thm} \label{lbg covering thm}
    对于 cpt 的 $E$ ,取任意一个 open cover $\{U_{\alpha}\}$,都存在一个 $\epsilon > 0$ 使得对于 $E$ 中的任何一点 $p$,$B_{\epsilon}(p)$ 都被包含在某个 $U_{\alpha}$ 中.
\end{lemma}
\proof 这个定理其实非常直观。因为 compact 一定有 finite subcover. 既然是 finite 的,那么任意一个 cover set 和集合或其他 cover set 边缘擦边的地方一定是有限的。既然 $E$ 是 compact 的那么一定是 complete 的那么当然也是 closed 的,于是取这些 finite 个擦边的地方里面最擦边的地方作为 $\epsilon$ 就可以了.

\begin{remark}
    显然,metric space 中 \textbf{complete 蕴含 closed,ttl bdd 蕴含 bdd},而特别地, $\mathbb{R}^n$ 中的任意集合,closed 等价于 complete,ttl bdd 等价于 bdd.
\end{remark}

\begin{theorem}
    \textbf{complete 的 metric space 的任意 closed subset 一定也是 complete 的}. 显然.
\end{theorem}
complete 的 metric space 的任意 closed subset(我们知道 metric space 的任意子集也是 metric space,by induced metric.)一定也是 complete 的. 显然.

\begin{remark}
    我们在 lec 1 中讲到  $l^{\infty}(\mathbb{N})$ 中的单位球是 closed + bdd 但是不 compact. 由于这是一个 Banach space 它也是 complete 的,by 3;不过它的问题出在并不是 ttl bdd 的.
\end{remark}

   


\chapter{Differentiation}
\section{Differentiation on $\mathbb{R}^n$}
\subsection{Continuity in MS}
\begin{remark}
    continuity 的定义: \textbf{open set 的 preimage 只能是 open set;等价于 closed set 的 preimage 只能是 closed set}
\end{remark}


\begin{theorem}
任意\textbf{ topological space 之间的连续映射} $ f:X \rightarrow Y$,\textbf{compact set 的 image 一定 compact}.
\end{theorem}

\begin{theorem}
任意 \textbf{topological space 之间的连续映}射 $f:X \rightarrow Y$,如果 \textbf{X compact,那么这个映射是 uniformly ctn 的}.(逆命题不成立)
\end{theorem}

\begin{definition}
    \textbf{$l_p$ norm on vector spaces:}\\
$$||v||_p = (\sum_{i = 1}^n |v_i|^p)^{1\over p}$$
\end{definition}

\begin{theorem}
    在\textbf{有限维 real vector space} (真包含 $\mathbb{R}^n$)中,\textbf{任意 $l_p$ norm 都等价}(无限维则不等).\\
    具体即:
    \begin{enumerate}
        \item 对于任意 $p_1, p_2$,都存在 $C_1, C_2$ 使得对于所有 $v\in V$:
        $$
        C_1||v||_{p_1} \leq ||v||_{p_2} \leq C_2||v||_{p_1}
        $$
        \item 任意 $l_p$ norm (通过差 norm 作为 metric) 定义出的\textbf{有限维 topological vector space 都是等价的},于是它们在 continuity, differentiability 以及 convergence 等等概念上的形式都是一样的.
    \end{enumerate}
\end{theorem}

\subsection{Derivative}
\noindent 阐明了这件事情之后我们看欧式空间之间 differentiability 的定义:一个 $f: U \subset \mathbb{R}^n \rightarrow \mathbb{R}^m$ 在某个点 $x_0$ 上 diffble 的定义就是 \textbf{在 $x_0$ 上可以被某个 $Hom(\mathbb{R}^n, \mathbb{R}^m)$ 上的某个 Linear map 一阶近似,并且这个近似在 x 点附近 converge by norm(误差的下降速度 $>>$ 邻域的收缩速度)}
即:
\begin{definition}{Derivative} \label{derivartive}
    $$
    Df(x_0) \in \mathbb{R}^{m \times n} \text{ is the matrix } A \text{ such that } \lim_{||h|| \rightarrow 0} \frac{||f(x_0 + t) - f(x_0) - Ah||}{||h||} = 0
    $$
    注意这里的 $Df$ operator 是一个函数:
    $$
    Df : E \rightarrow Hom(R^n, R^m)
    $$
\end{definition}

我们可以由原函数和 derivative induce 出一个 remainder 函数,来表示在某点处 derivative 的 linear approximation 的误差程度随微小量 $h$ 变化的变化. 

\begin{definition}
我们定义 remainder function: (用来做分析时使用)
\begin{align}
    r_{x_0} : \mathbb{R}^n &  \rightarrow \mathbb{R}^m \\
    h & \mapsto f(x_0 + h) - f(x_0) - Df(x_0) \cdot h
\end{align}
\end{definition}

\noindent 对于任意的 $x_0$,总是有 $||\frac{r_{x_0}(h)}{h}|| \rightarrow 0$ as $h \rightarrow 0$,因而 $r(h)$ 是 $o(h)$ 的.

\begin{theorem}{$\mathbb{R}^n \rightarrow \mathbb{R}^m$ is similar to $\mathbb{R}^n \rightarrow \mathbb{R}$} \label{$f$ ctn/diff iff $f_i$ ctn/diffble for all $i$}
    For $$f: \mathbb{R}^n \rightarrow \mathbb{R}^m = \begin{pmatrix}
        f_1 \\ f_2 \\ \vdots \\ f_m
    \end{pmatrix}
    $$ we always have
    $f$ ctn/diff iff $f_i$ ctn/diffble for all $i$
\end{theorem}
意思是在证明可导/连续时我们总是可以只考虑 $m=1$ 的情况. 这件事非常有用.

\section{Directional Derivative}
\begin{definition}
    固定一个 $u \in \mathbb{R}^n$,define 
    \begin{align}
        D_u f(X_0) : \mathbb{R}^n & \rightarrow \mathbb{R}^m\\
        x & \mapsto \lim_{t\rightarrow 0, t\in \mathbb{R}} \frac{f(x + tu) - f(x)}{t}
    \end{align}

        Notice that:
        $$
        D_u f(x_0) = \frac{d}{dt}\bigg| _{t=0} f(x_0 + tu)
        $$
\end{definition}


\begin{proposition}{Intuition of Directional Derivative}
我们发现,实际上 $D_v f$ 和 $f$ 是来自同一个 function space $\mathcal{F}(\mathbb{R}^n, \mathbb{R}^m)$ 的.\\
实际上,$D_v f$ 描述的对于每个 $x \in dom(f)$,在点 $x$ 处 $f$ 在 $v$ 这个方向上的变化率. 这里 $f$ 才是变量,$v$ 是固定的。\\
$D_v f$ 对每个(存在该 directional derivative)的 $x \in dom(f)$ 都给出了一个 $\mathbb{R}^m$ 数值,对应 $f(x)$ 的值给出了 $f$ 在方向 $v$ 上的变化率 $D_v f(x)$. 所以整个 $D_v f$ 就是 $f$ 对应的一个变化率函数.
\end{proposition}

\begin{remark}
    $\mathbb{R}^n$ 是一个 vector space. 我们自然想到,如果知道每个 $e_i$ 方向上 $f$ 对应的变化率函数 $D_{e_i} f$,那么就知道了整体 $f$ 在各个方向上的变化率函数. 我们对这个函数投一个点 $x = c_1 e_1 + \cdots + c_n e_n \in dom(f)$:
    $$
    Df(x_0)(x) = \sum_{i=1}^n c_i D_{e_i} f(x_0) 
    $$
    就可以得到这点上 $Df(x_0)$ 的作用.
\end{remark}

\noindent 我们有一个专门的称呼来指代 $D_{e_i} f$
\begin{definition}{partial derivative} \label{partial derivative}
    我们使用 $\frac{\partial }{ \partial x_i} f$ 来表示 $D_{e_i} f$
\end{definition}

接着刚才的 remark,我们发现
\begin{theorem}
    如果 $f $ 在 $x_0 \in dom(f) $ 处 diffble,那么它在 $x_0$ 处的 derivative 为:
    \begin{align}
        Df(x_0) 
        & = \begin{pmatrix}
        \frac{\partial f_1}{\partial x_1} (x_0)& \frac{\partial f_1}{\partial x_2}(x_0) & \cdots & \frac{\partial f_1}{\partial x_n}(x_0) \\
        \frac{\partial f_2}{\partial x_1} (x_0)& \frac{\partial f_2}{\partial x_2}(x_0) & \cdots & \frac{\partial f_2}{\partial x_n}(x_0) \\
        \vdots & \vdots & \ddots & \vdots \\
        \frac{\partial f_m}{\partial x_1}(x_0) & \frac{\partial f_m}{\partial x_2} (x_0)& \cdots & \frac{\partial f_m}{\partial x_n}(x_0)
        \end{pmatrix}
    \end{align}

    其中每一列就是 $D_{e_i} f(x_0) = \frac{\partial}{\partial x_i} f(x_0)$. 因而对于每个输入 $ c \in \mathbb{R}^n$,我们得到的输出都是 $\sum_{i=1}^n c_i \frac{\partial}{\partial x_i} f(x_0) \in \mathbb{R}^m$.\\
    
    \begin{align}
        Df(x_0) (c)
        & = \begin{pmatrix}
        \sum_{i=1}^n c_i \frac{\partial}{\partial x_i} f_1(c) \\
        \sum_{i=1}^n c_i \frac{\partial}{\partial x_i} f_2(c) \\
        \vdots  \\
        \sum_{i=1}^n c_i \frac{\partial}{\partial x_i} f_n(c)
        \end{pmatrix}
    \end{align}
我们称这个 matrix 为 $f$ 的 \textbf{Jacobian matrix at $x_0$}
\end{theorem}

\begin{remark}
    $$
        Df: A \subset \mathbb{R}^n \rightarrow \text{Hom}(\mathbb{R}^n, \mathbb{R}^m) 
    $$
    $$
        x \mapsto 
        \begin{pmatrix}
        \frac{\partial f_1}{\partial x_1} (x)& \frac{\partial f_1}{\partial x_2}(x) & \cdots & \frac{\partial f_1}{\partial x_n}(x) \\
        \frac{\partial f_2}{\partial x_1} (x)& \frac{\partial f_2}{\partial x_2}(x) & \cdots & \frac{\partial f_2}{\partial x_n}(x) \\
        \vdots & \vdots & \ddots & \vdots \\
        \frac{\partial f_m}{\partial x_1}(x) & \frac{\partial f_m}{\partial x_2} (x)& \cdots & \frac{\partial f_m}{\partial x_n}(x)
        \end{pmatrix}
    $$
    \textbf{显然,如果在 $A$ 上每个 partial derivative 都是 continuous 的,那么这个 $Df$ 也是 continuous 的(if we define norm on $\text{Hom}(\mathbb{R}^n, \mathbb{R}^m) $}
\end{remark}

\begin{definition}{higher-order partial derivatives}
    我们知道 $f: \mathbb{R}^n \rightarrow \mathbb{R}^m$ 的任意 partial derivative ${\partial \over \partial x_i} f: \mathbb{R}^n \rightarrow \mathbb{R}^m$ 仍然是一个 $\mathbb{R}^n \rightarrow \mathbb{R}^m$ 的函数. 我们再对这个函数求 $e_j$ 方向上的 partial derivatve,可以得到另一个 ${\partial^2 \over \partial x_j \partial x_i} f: \mathbb{R}^n \rightarrow \mathbb{R}^m$.\\
    recursively define: 
    $$
    {\partial^d \over \partial x_{k_d}\partial x_{k_{d-1}}\cdots \partial x_{k_1}} = \frac{\partial }{\partial x_{k_d}}[{\partial^{d-1} \over \partial x_{k_{d-1}}\partial x_{k_{d-2}}\cdots \partial x_{k_1}}]
    $$
    为一个 $d^\text{th}$-oder 的 partial derivative of $f$
\end{definition}
\begin{remark}
    只有 partial derivative 是有多阶的,而 derivative $D$ 也急速hi全导数是没有多阶的,因为它本身就是一个 linear map. 因为 $f: \mathbb{R}^n \rightarrow \mathbb{R}^m$ 的任意阶段偏导数也是 $\mathbb{R}^n \rightarrow \mathbb{R}^m$ 的,但是全导数是一个从元素到线性变换的函数.\\
    我们平时使用的单变量real-valued函数的高阶导数是 partial derivative 的特殊情况,这是在 $\mathbb{R} \rightarrow \mathbb{R}$ 上的函数只有一个变量,全导数和偏导数是同一个函数.
\end{remark}

\section{$C^r$ class}
\begin{definition}{$C^r$ class}
    如果在 $A \subset \text{dom}(f)$ 上, $f$ 的所有 $r^\text{th}$-order partial derivatives 都 \textbf{exist 并且 ctn},则称 $f \in C^r(A)$.
    \\如果对于任意 $r \in \mathbb{N}$ 都有 $f \in C^r(A)$,则称 $f \in C^{\infty}$
\end{definition}


\begin{remark}
    $f \in C^{r+1}(A) \Leftrightarrow $ all partials of $f$ are in $C^r(A)$
\end{remark}



    如果 f 在点 $x$ 附近 being $C^1$: $f$ 在点 $x$ 附近,每个 $e_i$ 切向量上的变化率函数都是连续的. 所以我们有一种直觉:$f$ 在这个点上是可以被一个 linear map 近似的.

\begin{theorem}{Sufficient condition for $f$ differentiable at some point}
    如果 $f: \mathbb{R}^n \rightarrow \mathbb{R}^m$ 在点 $x_0$ 附近是 $C^1$ 的,那么它在 $x_0$ 上是 differentiable 的 (即 $Df(x_0)$ exists.\\
    (Note: 此时在 $x_0$ 附近 $Df: A \subset \mathbb{R}^n \rightarrow \text{Hom}(\mathbb{R}^n, \mathbb{R}^m) $ 也是 continuous 的.)
\end{theorem}
\begin{proof}
见L05.
\end{proof}

\begin{remark}
    因而如果 $f \in C^1(dom(f))$, $f$ 就是 differentiable 的函数.
\end{remark}


\begin{theorem}{Mixed partials}
    If $f$ is $C^2$ 则任取 $x \in \text{dom}(f)$,$f$ 在 $x$ 处的任意 $2^\text{nd}$-order partial 可以更换次序. 即:
    $$
    {\partial^2 \over \partial x_{j_1} \partial x_{j_2}}f(x_0) = {\partial^2 \over \partial x_{j_2} \partial x_{j_1}} f(x)
    $$
\end{theorem}
\begin{proof}
    见 L06.\\
    对于这种 multivariable 的关于 diffbility/ ctnity 的证明,我们总是应该想到把它先等价为 $\mathbb{R}^n \rightarrow \mathbb{R}$,然后再对 $n$ 个变量进行逐个处理,运用 single variable analysis 的各种 mean value theorem 来处理.
\end{proof}


\begin{corollary}{Mixed partials}\label{Mixed partials}
    自然可以得到:如果 $f \in C^r(A)$,那么 $f$ 的任意 $r$ 阶及以下阶 partial derivative 都可以换序. 即: $\forall m \leq r$ 以及 $\forall \pi \in S_m$:
    $$
     {\partial^m \over \partial x_{k_1}\partial x_{k_{2}}\cdots \partial x_{k_m}} = {\partial^m \over \partial x_{k_{\pi(1)}}\partial x_{k_{\pi(2)}}\cdots \partial x_{k_{\pi(m)}}}
    $$
    因而如果 $f\in C^\infty (A)$ 那么所有的 partial derivative 都可以换序.
\end{corollary}


\section{Chain Rule, Product Rule and Taylor's theorem}
\begin{definition}{multi-index notation} \label{multi-index}
一个 multi-index 是一个 $n-$tuple,其中每个 entry 都是一个 non-neg integer.\\
我们使用以下四个 notation 来进行简写:\\
$$
|\alpha| = \sum_{j = 1}^n \alpha_j
$$
$$
\alpha! = \prod_{i = 1}^n (\alpha_j !)
$$
$$
x^{\alpha} = {x_1}^{\alpha_1}{x_2}^{\alpha_2} \cdots {x_n}^{\alpha_n}
$$
$$
\partial ^ \alpha f ={\partial^{|\alpha|} \over (\partial {x_1})^{\alpha_1} (\partial {x_2})^{\alpha_2} \cdots (\partial {x_n})^{\alpha^n}} f  
$$
\end{definition}
这个 notation 的引入会使得很多定理的表述简洁很多.
\begin{example}
$$
\partial ^{(2,1)} f = \frac{\partial^3}{\partial x_1 \partial x_2 \partial x_3 } f
$$
$$
(3,3,0)! = 3!3!0! = 36
$$
\end{example}


\subsection{Chain Rule}
\noindent Recall chain rule in single variable analysis:
$$f: A \subset \mathbb{R} \rightarrow B \subset \mathbb{R} \;\; \text{, } g: B \rightarrow \mathbb{R} 
$$
with A, B open; $f$ is diffble at some point $x$ and $g$ is diffble at $f(x)$.\\
Then we have: 
$$
\frac{d}{dx} (g \circ f) (x) = g'(f(x)) f'(x)
$$

现在我们有 multi 的版本:
\begin{theorem}
    如果 $$f: A \subset \mathbb{R}^n \rightarrow B \subset \mathbb{R}^m \;\; \text{, } g: B \rightarrow \mathbb{R}^p 
$$
with A, B open; $f$ is diffble at some point $x$ and $g$ is diffble at $f(x)$.\\
Then we have:
$$
D (g \circ f) (x) =Dg(f(x))\, Df(x)
$$
\end{theorem}
和 single varable 的表述基本一模一样.
\begin{proof}
    见 L07. 证明思路即取 $f,g, g\circ f$ 的 remainder function 并发现在 $R_f, R_g$ 趋向 0 时 $R_{g\circ f}$ 也趋向 0.
\end{proof}

\subsection{Multinomial Theorem}
\begin{theorem}{Multinomial Thm} \label{Multinomial Thm}
$\forall (x_1, x_2, \cdots, x_n) \in \mathbb{R}^n$ 以及任意 $k \in \mathbb{Z}_{\geq 0}$ we have:
$$
{(x_1 + \cdots + x_n)}^k  = \sum_{|\alpha| = k} \frac{k!}{\alpha !} x^\alpha 
$$
\end{theorem}
这是一个 combinatorics 的 proposition. 
\begin{proof}
    (Informal proof, but sufficient to show it.)\\
    ${(x_1 + \cdots + x_n)}^k$ 每项就是每次在 $\{x_1, x_2, \cdots, x_n\}$ 中取一个元素,取 $k$ 次得到,一共有 $\sum_{|\alpha| = k} \frac{k!}{\alpha !} = k!$ 个项,每一项都是一个 $x^\alpha$ where  $|\alpha| = k$.\\
    每个 $x^\alpha$ 会被取到重复次,这个重复的数量为 ${k! \over \alpha !}$\\
     因为 $k! \over \alpha !$ 就是把一个 size 为 $k$ 的 set 分成 size 分别是 $\alpha_1, \cdots, \alpha_n$ 的子集的方法.\\
     推导: 
$$
\# =  C^k_{\alpha_1} \times  C^{k-\alpha_1}_{\alpha_2}  \times C^{k-\alpha_1 - \alpha_2}_{\alpha_3} \times\cdots \times C^{k - \sum_{i=1}^{n-1} \alpha _i}_{\alpha_n} = {k! \over \alpha_1! \alpha_2 ! \cdots \alpha_n !}
$$
    (对于 for example:  $\alpha = \{1,0,2,2,0\}$ 即 $x_1 {x_3}^2 {x_4}^2$ 这一项,会被取到 ${k! \over \alpha !} = \frac{5!}{2!2!}$ 次. 这个次数等于$ 5 \times C^4_2 \times C^2_2 $ )
    \\ formal 的证明即 by induction,使用 binomial thm 即可.
\end{proof}

\subsection{Product Rule}

\begin{lemma}{在 $\mathbb{R} \rightarrow \mathbb{R}$ 上的函数的 product rule}
令 $f_1, \cdots f_m: \mathbb{R} \rightarrow \mathbb{R}$

$$
\frac{d^k {(f_1,\cdots f_m)}}{dx^k} = \sum_{|\alpha| = k} \frac{k!}{\alpha !} (\frac{d^{\alpha_1} f_1}{dx^{\alpha_1}}) \cdots (\frac{d^{\alpha_m} f_m}{dx^{\alpha_m}})
$$

\end{lemma}

\begin{note}
    特别地,$$
    \frac{d^k}{dx^k} (fg) = \sum_{a+b = k} \frac{k!}{a!b!} (\frac{d^a}{dx^a}f) (\frac{d^b}{dx^b} g)
    $$
    这个式子也等价于 $$
    \sum_{a = 0}^k \frac{k!}{a!(k-a)!} (\frac{d^a}{dx^a}f) (\frac{d^{k-a}}{dx^{k-a}} g)
    $$
\end{note}

\begin{proof}
    见 HW5 G. \\
\end{proof}

\begin{theorem}{product rule} \label{product rule}
$\forall \alpha \in \mathbb{Z}_{\geq 0}^n$,$f,g: \mathbb{R}^n \rightarrow \mathbb{R}$, 如果 $f,g \in C^{|\alpha|} (\mathbb{R}^n)$, then:
$$
\partial^{\alpha} (fg) = \sum_{\beta + \gamma = \alpha} \frac{\alpha !}{\beta ! \gamma !} (\partial ^ \beta f) (\partial ^ \gamma g)
$$
\end{theorem}
\begin{proof}
    由 lemma 由归纳得到,见 L08.
\end{proof}
\begin{note}
    实际上这里 f,g 是 $\mathbb{R}^n \rightarrow \mathbb{R}$ 的函数并不是重要的,重要的是 $\alpha$ 和几个变量有关系。如果 $f,g:\mathbb{R}^{100} \rightarrow \mathbb{R} $,$\alpha \in \mathbb{Z}_{\geq 0} ^{100}$ 但是只有一个 entry 不是 0,那么我们可以看作 $\alpha \in \mathbb{Z}_{\geq 0}^1$,使用 $\mathbb{Z}_{\geq 0}^1$ 上的 Product rule.
\end{note}


\begin{lemma}{Taylor Thm in $\mathbb{R}$}
如果 $f: I \subseteq \mathbb{R} \rightarrow \mathbb{R} \in C^k(I)$,并且 f 是 $(k+1)$ - diffble 的.\\
(即,$f$ 几乎是 $C^{k+1}$ 的,但是并不需要 k+1 阶导数也连续,条件更宽)\\
任取 $a \in I$,我们定义 \textbf{the $k^\text{th}$ Taylor polynomial centered at $a$}:
$$
T_{a,k}(x) = f(a) + \sum_{j=1}^k \frac{f^{(j)}(a)}{j!} (x-a)^j
$$
Then: 这个 $T_{a,k}(x)$ 是 \textbf{best $k^\text{th}$-order polynomial approximation centered at $a$}: $$
f(x) - T_{a,k}(x) =  R_{a,k} (x) = \frac{f^{(k+1)}(c_x)}{(k+1)!} (x-a)^{k+1} = o(|x-a|^k)
$$
Note: 这里的 remainder 和 a,k 都有关系. 是:
对于每个 $x \in I$,$c_x \in [a,x]$ 是由 Cauchy MVT 决定的一个中间值.
\end{lemma}

\begin{remark}
   但是这仍然并不能说明如果 $f \in C^{\infty}(I)$,就一定有 $f(x) = T_{a, \infty}(x), \forall x,a \in I$. \\
   Note:  $C^\omega (\mathbb{R}) \subset C^\infty(\mathbb{R}).$\\
    一个光滑函数 $f$ 在 $I$ 上解析的条件是在 $I$ 上这个余项函数是处处收敛到 0 的.
    具体: $
    \forall \text{compact } K \subseteq I, \; \exists m \in \mathbb{R} \text { s.t. } $
    $$
    \bigg |\frac{d^j}{dx^j} f(x) \bigg | \leq m^{k+1} j! 
    $$
    for all $x \in K$ and for all order $j$.
\end{remark}

\begin{proof}
    具体见 L08.\\
    思路:Recall that Rolle MVT $\rightarrow$ Lagrange MVT $\rightarrow$ Cauchy's MVT\\
    我们使用 Cauchy's MVT apply to 余项来证明 Taylor's Thm.
\end{proof}


\begin{theorem}{Taylor's Thm}\label{Taylor's Thm}
    
\end{theorem}











\section{Inverse function Theorem}
\begin{theorem}{Inverse function Theorem} \label{IFT}
    如果 $f: A \subseteq \mathbb{R}^n \rightarrow \mathbb{R}^m$ 是 $C^r$ 
 ($r \geq 1$)的,对于任意 $x_0 \in A$ s.t. $\det{Df(x_0)} \not= 0$,我们都有:
    \begin{enumerate}
        \item 存在某个 $x_0$ 的 nbh $U$ 以及 $f(x_0)$ 的 nbh $V$ 使得 $f(U) = V$ 且 $f|_U : U \rightarrow V$ 是 bijective 的. (即 $f$ 在 $x_0$ 附近是 \textbf{locally bijective }的).
        \item $f|_U$ 的 inverse function $g: V \rightarrow U$ 也是 $C^r$ 的,并且对于 $U$ 中任意 $x$,都有 $Dg(f(x)) = (Df(x))^{-1}$ \\ 
        (这是显然的,因为 $I_n = D (g \circ f) (x) = Dg(f(x))Df(x)$)
    \end{enumerate}
\end{theorem}
\begin{remark}
    比较直观。我们知道如果 $f \in C^1(A)$,意味着 Jacobian matrix 中每一项都是连续的函数,那么\textbf{整个 $Df: \mathbb{R}^n \rightarrow \mathbb{R}^{n \times n}$ 在 $x_0$ 附近也是连续的};并且由于 \textbf{$\det :\mathbb{R}^{n \times n} \rightarrow \mathbb{R}$ determinant 函数是连续的},我们知道整个 $Df$ 在 $x_0$ \textbf{locally 都是 non-sigular 的}. \\\\
    我们可以类比 single variable 情况下:如果 $C_1$ 的函数 $f$ 的导函数 $f'$ 在某处不是 0,那么在这一块地方附近 $f$ 都 monotonic. 这是因为 $f'$ 连续,所以这附近 $f'$ 都非 0 且同向.\\\\
    IFT 告诉我们的是:这个性质可以推广到多元函数,by: 单元函数导数值不为 0 扩展到多元函数导数的 det 不为0.\\
    并且 IFT 还多告诉我们的一件事情是: 这个 local inverse function 会保留 $C^r$ 的性质.\\
\end{remark}

\begin{lemma}
    对于任意的 matrix $C \in \mathbb{R}^{m \times n}$,并任取 $v \in \mathbb{R}^n$,我们都有:
    $$
    |Cx| \leq ||C|| \cdot |x|
    $$
    并且: 
    $$
||C||\leq nm \max_{1 \leq i \leq m, 1\leq j \leq n} |C_{ij}|
$$
where we take the sup norm of a linear map:
$$
||C|| = \sup_{x \in \mathbb{R}^n, |x| = 1} |Cx|
$$
\end{lemma}


\begin{lemma}
    如果 $f: A \rightarrow \mathbb{R}^n$ 是 $C^1$ 的,且 $Df(a)$ non-singular,即 $\det(Df(a)) \not = 0$,那么存在某个 $a$ 的 nbh $B_{\delta} (a)$ 使得对于任意 $x,y \in B_{\delta}(a)$,都有:
    $$
    |f(x_0) - f(x_1)| \geq \alpha |x_0 - x_1|
    $$
\end{lemma}
\begin{note}
 \textbf{这是一个比 locally injective 更强的条件.}
\end{note}







\section{Implicit Function Thm}












\chapter{Integration}
\section{Riemann Integrability over a box}



\section{Lebesgue's Characterization of Riemann integrable functions over a box}




\section{Fubini's Thm}
\begin{theorem}{Fubini's Thm}\label{Fubini's Thm}
    Conditions:\\
    (1) Let $Q = A \times B$, 其中 $A \subseteq \bR^k$, $B \subseteq \bR^l$ 都是 boxes.\\
    (2) Let $f: Q \rightarrow \bR$ be bounded.\\
    
    Statement: \\
    If $f$ Riem intable over $Q$, 那么 
    $$
    \ul{I}: x \mapsto \underline{\int_B} f(x,y) dy
    $$
    以及 
    $$
    \ol{I}:  x \mapsto \overline{\int_B} f(x,y) dy
    $$
    都是 \textbf{Riem intble over A} 的,并且有 

    $$
    \int_Q f = \int_A \underline{\int_B} f(x,y) \,dy \,dx = \int_A \overline{\int_B} f(x,y) \,dy \,dx
    $$
    
\end{theorem}
\begin{proof}
    一个 partition $P$ 一定可以分为 $P = (P_A, P_B)$, 它的每个 subbox $R = R_A \times R_B$.\\
    对于一个 $R$, 任取 $x_0 \in R_A$, 我们都有 $$m_R(f) \leq m_{R_B} f(x_0, \cdot)$$\\
    Summing over all $R_B \in P_B$, 我们有: 任取 $x_0 \in R_A$, 
    $$
    \sum_{R_B \in P_B} m_{R_A \times R_B} (f) v(R_B) \leq \sum_{R_B \in P_B} m_{R_B}(f(x_0, \cdot)) v(R_B) = L(f(x,\cdot), P_B) \leq I(x)
    $$
    Taking the inf over all possible $x_0$,
    $$
     \sum_{R_B \in P_B} m_{R_A \times R_B} f v(R_B) \leq m_{R_A}(\ul{I})
    $$
    而后 sum over all possible $R_A$, 得到:
    $$
    L(f,P) \leq L(\ul{I}, P_A)
    $$
    于是我们可以把 $L(\ul{I}, P_A), U(\ul{I}, P_A) $ bound 在 $L(f,P), U(f,P)$ 之间. Have four similar arguments. This gives the result.
\end{proof}



\subsection{Corollaries}

\begin{corollary}{Corollary 1}
    Conditions:和 Fubini Thm 相同\\
    Statement:\\
    $\int_B f(x,y) dy \;$  exists a.e. on $A$.
\end{corollary}
\begin{remark}
    即 $m (\{ x |  y \mapsto f(x,y) \text{ not intble over B} \}) = 0$ 
\end{remark}



\begin{corollary}{Corollary 2}{Natural Corollary}
Conditions: $Q \subseteq \bR ^ n$ 是一个 box.

Statement: 如果 $f: Q \rightarrow \bR$ ctn, 那么
$$
\int_Q f = \int^{b_1}_{a_1} \int^{b_2}_{a_2} \cdots \int^{b_n}_{a_n} f \; d x_n \cdots d x_1
$$
\end{corollary}

\begin{proof}
    平凡推论. ctn 函数, 不论 $Q$ 如何分维度 boxes, 都是 intble 的. 所以 by induction 可以一个一个维度积分.
\end{proof}






\section{Riemann Integrability over bounded sets}

\begin{definition}{integral over bounded sets}
    Conditions:\\
    (1) $S \sub \bR^n$ bdd, \\
    (2) $f: S \rightarrow \bR$ bdd\\
    Define: 
    $$
    f_S (x) = \begin{cases}
        f(x) \; \text{, if } x \in S \\
        0 \; \text{, else}
    \end{cases}
   $$
   Now we define \textbf{the integral of $f$ over $S$}: 
   $$
   \int_S f(x) dx = \int_Q f_S (x) dx
   $$
   p.t that $\int_Q f_S (x) dx$ exists.
\end{definition}
\begin{remark}
    我们把 integral 推广到任意 bounded set 上, 通过取一个 $S$ 的 covering box, 并取一个辅助函数, 在 $S$ 上面是 $f$, 在 $S$ 外面是 $0$.\\
    我们容易提出这一问题: 这个定义成立的前提是, 我们必须确定, 取不同的 covering boxes 得到的这个新函数的积分都是相同的. 下面这个 Lemma 表明了这一点. 
\end{remark}

\subsection{well-definedness of Riem integral over bdd set}
\begin{lemma}{The reason why our def of integral over bounded sets is fine:}
    Conditions:\\
    (1) $Q, Q' \sub \bR^n$ bdd.\\
    (2) $f: \bR^n \rar \bR$ supported only on $Q \intsec Q'$.\\
    Statement:\\
    (1) $f$ intble over $Q$ $\eq$ $f$ intble over $Q'$ \\
    并且 (2)
    $$
    \int_Q f = \int_{Q'} f 
    $$
\end{lemma}
\begin{proof}
    It suffices to prove it for $Q \sub Q'$. 因而只要证明了这一 case, 我们对于 $Q \not \sub Q'$ 的 case 也可以选取 $Q''$ s.t. $Q, Q' \sub Q$, 得到 $\int_Q f = \int_{Q''}f$, $\int_{Q'} f = \int_{Q''} f$.\\
    因而易证. partition 也可以通过包含构造. 多出的分割点无所谓, 因为在上面 m(f) 都是 0.
\end{proof}


\subsection{properties of Riem integral over bdd sets}
\begin{theorem}{properties of Riem integral over bdd sets}
    Conditions:\\
    (1) $S \sub \bR^n$ be bdd.\\
    (2) $f,g: S\rar \bR$ intble.\\
    Statement:
    \begin{enumerate}
        \item \textbf{Linearity of integral and intbility}: 
        $\forall c \in \bR$, $f + cg$ is also intble, 并且 
        $$ \int_S (f + cg) = \int_S f + c \int_S g$$
        \item 
        $M,m: S \rar \BR $ defined by 
        $$
        M: x \mapsto \max{(f(x), g(x))}
        $$
        $$
        m: x \mapsto \min{(f(x), g(x))}
        $$
        也是 intble 的.

        \item 
        $$f \leq g \; \forall x \implies \int_S f \leq \int_S g$$

        \item 
        $|f|$ 也是 intble 的, 并且 
        $$
        |\int_S f | \leq \int_S |f|
        $$

        \item \textbf{Monotonicity}: Let $T sub S$. If $f$ nonneg 且在 $T$ 上也 intble. Then:
        $$
        \int_T f \leq \int_S f
        $$

        \item \textbf{Additivity}: 
        $f$ intble on $S_1, S_2$ $\implies$ $f$ intble on $S_1 \union S_2$, $S_1 \intsec S_2$.\\
        并且有:
        $$
        \int_{S_1 \union S_2} = \int_{S_1} f + \int_{S_2} f - \int_{S_1 \intsec S_2} f
        $$

        \item 对于 \textbf{bdd 且 disjoint} $S_1, \cdots, S_m$, 有:
        $$
        \int_{\union_n S_n} f = \sum_n \int_{S_n} f
        $$
    \end{enumerate}
    
\end{theorem}


\subsection{Riem intble on bdd set 的等价条件}
\begin{remark}
    我们知道, $f$ intble on $S$ 的条件其实即存在一个 $Q \supset S$ 使得 $m(D_{f_S}) = 0$.\\
    我们可以把它等价为更加直接的条件:
\end{remark}

\begin{theorem}{Characterization of Riem Intbility on bdd sets}\label{Characterization of Riem Intbility on bdd sets}
    Conditions:\\
    (1) $S \sub \bR^n$ be bdd.\\
    (2) $f,g: S\rar \bR$ intble.\\
    Define:\\
    (1) $D_f = \{x \in S | f \text{ not ctn at } x\}$\\
    (2) $E = \{ x_0 \in \partial S  | \displaystyle \lim_{x \rar x_0} f(x) \not = 0 \}$
    \\Then:
    $$
    f \text{ Riem intble } \displaystyle\eq m(D_f) = m(E) = 0
    $$
\end{theorem}
\begin{remark}
    不需要证明. 实际上, 取一个 covering box, 创造出来的新函数的不连续点集比起原来的不连续点集合, 多出来的点的就是这里的 $E$.
    $$
    D_{f_S} \backslash D_f = E
    $$
\end{remark}



\subsection{Jordan measure 和 integral 的关系}
\begin{theorem}{Jordan measurable 对于 integrability 的意义} \label{equivalent conditions for Jordan-mesurability}
    Condition: $S \sub \BR^n $ is bdd.\\
    Statement: TFAE.
    \begin{enumerate}
        \item $S$ is Jordan measurable.
        \item const function $f = 1$ 在 $S$ 上 Riem intble.
        \item  $m(\partial S) = 0$
        \item  $\ol{m_J}(\partial S) = 0 $
    \end{enumerate}
\end{theorem}
\begin{proof}
    1,3,4 等价 by IBL, 显然.\\
    证明 2, 3 等价: 考虑 indicator function of S, on 一个任意的包含 S 的 box.
\end{proof}

\begin{remark}
Define: 
$$
\chi_S (x) = \begin{cases}
    1 \; , x \in S \\
    0 \; , x \not\in S
\end{cases}
$$
注意到
$$
D_{\chi_S} = \partial S
$$
\end{remark}


\begin{corollary}{To compute Jordan measure}
    $S$ Jordan measurable $\implies$ 
    $$
    m_J (S) = \int_S 1 
    $$
\end{corollary}



\begin{corollary}
    $S$ is Jordan measurable $\eq$ all ctn function from $S \rar \bR$ are Riem intble on $S$.
\end{corollary}




\section{Extended Integrals}
\begin{definition}{All Jordan measurable sets on $\bR$}
    记录一些 notations.\\
    $$
    \cJ = \{ \text{all Jordan measurable sets on } \bR^n \}
    $$
    $$
    \cJ_c = \{ \text{all Jordan measurable compact sets on } \bR^n \}
    $$
\end{definition}


\begin{definition}{negative and positive parts of a function}
    $$
    f_+ (x) = \max(f(x), 0)
    $$
    $$
    f_- (x) = \max(-f(x), 0)
    $$
\end{definition}

\begin{note}
    $f_+, f_-$ are all non-negative.\\
    并且有
    $$
    f = f_+ - f_-
    $$
    $$
    |f| = f_+ + f_-
    $$
\end{note}


\begin{remark}
    如果 $f$ ctn, 那么一定有 $f_+, f_-$ 也 ctn.
\end{remark}


\begin{definition}{extended integral}
    Condition:\\
    (1) $A \sub \BR^d$ open.\\
    (2) $f:A\rar \bR $ ctn.\\
    Define: the extended integral of $f$ over $A$:
    Case 1, for non-neg $f$:
    $$
    \int_A f = \sup_{D \sub A, D\in \cJ_c} \int_D f
    $$
    Case 2, for normal $f$:
    $$
    \int_A f = \int_A f_+ + \int_A f_-
    $$
\end{definition}
\begin{remark}
    extended integral 的理念: 对于任意一个 open set 上的 real-valued 函数, 我们把它的积分定义为这个 open set 包含的所有 compact J-measurable 子集上的函数积分的上确界.\\\\
    这个 extended def 主要服务于定义域 unbounded 的函数. 我们的 ordinary integral 只 define 在了 bounded set 上, 而我们现在把它扩展到某些 bounded set 上.\\
    我们施加的额外条件是这个函数连续. 所以现在我们对任何 $\BR^n \rar \bR$ 的连续函数都可以尝试求其 extended integral 是否存在.\\\\
    而注意到两个问题:\\
    (1) 这个定义对于 bounded set 上的连续函数也成立. 因而对于 bounded set 上的连续函数, 我们现在有两个积分定义, $ord \int_A f$ 和 $ext \int_A f$. \textbf{我们需要 justify 这两个定义对于这样的函数是等价的.}\\
    (2) 这个 extended integral 不仅 extend 了积分定义 from bounded set to unbounded set, 并且也扩展了 integrability. 即: \textbf{有些函数可以不是 Riemann integrable 的, 但是 extended integral 却存在.}
\end{remark}

\begin{example}
    下面是一个不 Riemann integrable 但是却存在 extended integral 的函数.
    考虑 $A$ bdd, open, not J-measurable(rectifiable).\\
    于是 by \ref{equivalent conditions for Jordan-mesurability}, 我们知道 $f(x) = 1, x\in A$ 是 not Riem intble 的.\\
    但是, by def of extended integral, 它却存在 extended integral.
\end{example}

\paragraph*{}现在我们 justify 这两个定义对于这样的函数是等价的. 关键在于: 任何 bounded open set 都可以用 compact elementary set 来无限逼近.

\begin{lemma}{open set can be approximated by cpt set}
    任意 open $A \in \bR^n$, 都有 a seq of compact elementary sets $(C_i)$ s.t. 
    $$
    C_i \sub C_{i+1}^{\circ} \forall i
    $$
    并且
    $$
    A = \bigunion_{i=1}^\infty C_i
    $$
\end{lemma}

\begin{remark}
    我们在 IBL 还有过另外一个 statement: 一个 open set in $\bR^n$ 都是一个 ctbl union of almost disjoint boxes. 这两个 statement 都给出了用 ctbl collection of cpt sets/ boxes 来分解 open set 的方法.
\end{remark}

\begin{proof}
    
\end{proof}






\chapter{Change of Variable Thm}
\begin{theorem}{Change of Variable}
Conditons:\\
(1) $A,B \sub \bR^n$ open.\\
(2) $g:A \rar B$ 是 $C^1$ diffeomorphism.\\
(3) $f:B \rar \bR$ ctn.\\

Statement:
$f$ 在 $B$ 上 Riem intble 当且仅当 $f(g(x))  |\det Dg(x)) |$ 在 $A$ 上 Riem intble.\\
并且有:
$$
\int_A f(g(x)) |\det (Dg(x))| dx = \int_B f
$$
    
\end{theorem}
\pic[0.4]{assets/change.png}
\begin{remark}
    当 $g$ 是一个微分同胚时, $f$ 在 $g(A)$ 上的积分就等于 $|det(Dg)| f\circ g $ 在 $A$ 上的积分.\\
    我们记得 determinant of a matrix 在 $\bR^n$ 上的集合意义是在 apply 这个 linear trans 后, unit cube 扩张的倍数.\\
\end{remark}




\section{diffeo preserves meas-0}
\begin{theorem}{$C^1$ map 把零测集映射至零测集}
Conditions:\\
(1) $A \sub \bR^n$ open\\
(2) $g: A \rar \bR^n$ 是 $C^1$ 的\\
Statement:\\
$\forall E\sub A $ s.t. $m(E)= 0$, 都有 $m(g(E)) = 0$ 
\end{theorem}
\begin{proof}
\textbf{Claim 1}: 对于一个零测集, 任取 $\epsilon, \delta$, 都可以用 ctbl 个 width 小于等于 $\delta$ 的 cubes 来 cover 它, 并使得 cubes 的总体积 $< \epsilon$.\\
这个 statement 显然. 注意这里我们使用的是 cube 而不是 box, 也就是边长相同的 box. 这是一个 trivial corollary of 零测集的定义.\\
\textbf{Claim 2}:任意 closed cube $C$ 在 $C^1$ map 下的 image, 可以用一个边长是它的 $n\sup_{x \in C} ||Dg(x)||$ cube 来 cover. 这个 Claim 容易证明.\\
现在: WTS $g(E) = 0$.\\
回忆:\\
(1) ctbl 个零测集的测度和也是零测集.\\
(2) 我们可以用 a seq of containing cpt sets $\{C_k \}$ 来获得一个 open set.\\
令 $E_k := E \intsec C_k$, 我们有$E = \union E_k$.\\
因而 it suffices to show $m(g(E_k)) = 0$.\\
\textbf{Claim 3}: 每个 $E_k = C_k \intsec E$, 其 $g(E_k)$ 的 measure 都是 0\\
\textbf{Claim 3.1}: 对于每一个 $C_k$, 都存在 $\delta > 0$, 使得 $C_k$ 的 $\delta-$nbh 被包含在 $C_{k+1} ^ {\circ}$ 中. (这是一个关于每个 $C_k$ 和其 $C_{k+1}$ 之间的空隙的厚度的命题. 既然 $C_K \sub C_{k+1} ^ {\circ}$ 那么它们之间的距离存在一定厚度,)\\
\textbf{Collect things together:}
\begin{enumerate}
    \item 由于 $E$ 零测, 每个 $E_k$ 都是零测的. By Step 1, 我们可以用 ctbl 个 width $\leq \delta$ 的 cubes 来 cover $C_k$ 使得总体积 $< \frac{\epsilon}{(nM)^n}$. 
    \item 并且 by Claim 3.1, 这个 cover $\{D_i\}$是 within $C_{k+1} $ 的. 
    \item by claim 2, 我们可以 bound 每个 $D_i$ 的 image by 边长小于等于 $nM$ width($D_i$) 的 box, 于是所有 $D_i$ 的 image 的 volume sum bounded by $(nM)^n \epsilon /(nM)^n = \epsilon$. 于是 $m(g(E_k)) = 0$.
\end{enumerate}
\end{proof}

\begin{remark}
我觉得这里比较 insightful 的证明手段, 除去通过 operator norm 来 bound image 的大小之外, 还有就是 $C_k$ 的 cover by $\delta$-width cubes 是 within $C_{k+1}$ 的这一点. \\
我们有时候觉得 Euclidean metric 是一个很麻烦的东西, 因为它无法推断每个维度的信息. 有时候需要在每个维度上都分别去 bound 一个东西: 就比如这里, 我们需要 bound 的是一整个 open cover of $C_k$, 使得这个 cover within $C_{k+1}$.\\
这个时候我们可以混用 sup metric. 这里我们把 boxes 的 cover 转化为了 cubes 的 cover, 就是为了 bound 住这个 cover, 因为 cube 的每个维度都相同. 这里 Claim 3.1 先证明了一个 sup metric 的命题, 然后利用它来 bound 住 cube 的范围, 让 cube 不会出 $C_{k+1}$ 的边界: \textbf{by supmetric, 每一个维度都留了 $\delta$ 的空隙}. 从而它 bound 住一整个 cover 的 union. 
\end{remark}





\begin{corollary}{从小维度到大维度的 diffble map 的 image 零测}
    对于 $C^1$ 的 $f: A \sub \bR^b \rar \bR^m $($A$ open), 如果 $m > n$ (小维度映射到大维度), 那么一定有 
    $$
    m(f(A)) = 0
    $$
\end{corollary}
\begin{proof}
    考虑
    \begin{align}
        g: A \times \bR^{m-n} \sub \bR^m &\rar bR^m\\
        (a,b) & \mapsto f(a)
    \end{align}
    于是 $f(A) = g(A \times \{0\}$.
\end{proof}
\begin{remark}
    如果 $f$ 不是 $C^1$ 的而是仅仅只是连续,则不成立. 考虑 space-filling curve.\\
\end{remark}


\section{diffeo preserves topological behavior AND J-measbility}
\begin{theorem}{diffeomorphism preserves boundary and interior 以及 J-meas}
Conditions:\\
(1) $A,B \sub \bR^n$ open\\
(2) $g: A \rar B$ 是 diffeo\\
(3) $D \sub A$ cpt.\\
Statement:\\
(1) 
$$
g(D^{\circ}) = (g(D))^{\circ}
$$
$$
g(\partial D) = \partial(g(D))
$$
(2) 并且如果 $D$ J-meas, 则有 $g(D)$ J-meas.
\end{theorem}
\begin{remark}
    J-meas 的 set 的 boundary 的 L-meas 是 0, 但是 L-meas 的 set 的 boundary 的 L-meas 不一定.\\
    比如 $\bQ$ 是 L-measurable 的, 但是它的 boundary 是整个 $\bR$.\\
    这是因为 J-meas 是一个太过于严格的性质. 你必须用 elementary set 来逼近它. 
\end{remark}


\section{decompose diffeo into finite primitive diffeos}

\begin{definition}{primitive map}
对于 $A, B \in \bR^n$, 如
    如果 $h:A \sub\bR^n \rar B \sub\bR^n$ 中, 有任何一个 subfunction $h_i(x) = x$, 即 preserves the $i$-th coord, 则称这个 function 为一个 primitive map
\end{definition}


\begin{lemma}{linear/affine map can be decomposed into finite prmitives}
任何 linear map 以及 translation 都可以被 decompose into finite primitives.
\end{lemma}
\begin{proof}
    Linear map 可以分解为 elementary matrices.
    \pic[0.6]{assets/compose.jpg}
    \pic[0.6]{assets/compose2.jpg}
    至于 translation 则显然.
\end{proof}
\begin{remark}
    所以任意一个 affine linear map 都可以分解成 primitives.
\end{remark}


\begin{theorem}{Any Diffeomprphism can be decomposed into finite primitive diffeomorphisms}
Conditions:\\
(1) $A, B$ open in $\bR^n$.\\
(2) $g:A \rar B$ 为一个 diffeomorphism.\\
Statements:\\
对于任意 $a \in A$, 都存在 open nbh $U_0 \sub A$ of $a$, 以及 a finite seq. of \textbf{primitive diffeomorphisms}
$$
u_0 \xrar{h_1} u_1 \xrar{h_2} \cdots \xrar{h_k} u_k \sub B
$$
s.t. 
$$
h_k \circ h_{k-1} \circ \cdots \circ u_k  = g |_{u_0}
$$
\end{theorem}
\begin{proof}
    我们首先 prove $a = 0, g(a) = a, Dg(a) = 0$ 时的特殊情况.\\
    这一情况下, 定义:
    $$
    h (x) = (g_1(x), \cdots, g_{n-1}(x), x_n)
    $$
    $$
    k(y) = (y_1, \cdots, y_n, g_n(h^{-1}(y))
    $$
    使用两次 IFT 得到 $g$ 在 $a$ 附近等于 $k \circ h$
    \pic[0.6]{assets/diff.jpeg}

    For general case $g$: 我们发现, 总是可以通过 translate, apply $g$, 再 apply $ (Dg(a))^{-1}$ 的方式获得一个新的 map, 使得它满足 special case.\\
    新 map 可分解, linear map 可分解, translate 可分解. 于是整个都可以分解.
\end{proof}




\section{Partition of Unity}

\subsection{bump function lemma AND cube partition lemma}

\begin{lemma}{bump function Lemma: 构造 supported only on a box 的光滑函数}
    对于任意 box $B \sub \bR^n$, 都存在 $C^{\infty}$ 的 non-neg 函数 $\phi: \bR^n \rar \bR$ 使得 $\supp(\phi) = \ol{Q}$
\end{lemma}
\begin{proof}
    构造:
    令
    $$
    f(x) = 
    \begin{cases}
    \exp(\frac{-1}{x}) \; , x >0 \\
    0 \; , x \leq 0
    \end{cases}
    $$
    再 define:
    $$
    g(x) = f(x)f(1-x)
    $$
    注意到这里的 $g$ 是一个仅在 $(0,1)$ 上 supported 的函数.\\
    通过放缩与多维度叠加, 就可以得到对任意 box 的构造:
    对于 $Q = [a_1, b_1] \times \cdots \times [a_n, b_n]$ we define
    $$
    \phi(x) := \prod_{i = 1}^n g(\frac{x_1 - a_i}{b_i - a_i})
    $$
    \pic[0.5]{assets/smooth.png}
\end{proof}




\begin{lemma}{Cube Partition Lemma: 分解 open set 为重叠程度是 locally finite 的 cubes}
Let $\sA$ be a collection of open sets in $\bR^n$. Let $A  = \union_{S \in \sA} S$.\\
There exists a seq of closed boxes(actually cubes) $Q_1, Q_2, \cdots, $ s.t.
\begin{enumerate}
    \item $A \sub \union_{i} Q_i ^{\circ}$
    \item Each $Q_i$ is containted in some element of $\sA$ (因而, 实际上 $A  = \union_{i} Q_i ^{\circ}$)
    \item \textbf{(Local Finiteness)} Every point of $A$ has some nbh that only intersects finitely many $Q_i$
\end{enumerate}
\end{lemma}
\begin{remark}
这个 theorem 又是另一个用 ctbl 个 closed boxs 来得到一个 open set 的 statement. Recall 我们已经有:\\
(1) open set 可以分解成 ctbl 个 almost disjoint closed boxes 的 union.\\
(2) 可以用 a seq. of cpt sets 来逼近一个 open set.\\
现在我们有第三个 statement:\\
(3) 即便是 unctbl 个 open sets 的 union(仍然是 open 的), 我们也可以找到一个 ctbl decomposition of closed cubes, 使得其中每个元素都完全包含在一个 open element 里; 并且 closed cubes 之间的排布不会太过紧密 (任意点都至多只 locally intersect 有限个 closed cubes.)\\
\end{remark}

\begin{proof}
    首先 take 我们的 seq of containing cpt sets of $A$, 写作 $\{D_i\}$.\\
    我们这次做一件我想做很久了的事情: 我们 take:
    $$
    B_i = D_i - D_{i-1}^{\circ}
    $$ for each $D_i$. (对于 $i\leq 0$ 取 $D_i$ 是空集.)\\
    于是把相互包含的 cpt sets $\{D_i\}$ 分解成了 almost disjoint 的 cpt sets $\{B_i\}$. \\
    我们 recall 上次的证明: 对于每一个 $C_k$, 都存在 $\delta > 0$, 使得 $C_k$ 的 $\delta-$nbh 被包含在 $C_{k+1} ^ {\circ}$ 中. (这是一个关于每个 $C_k$ 和其 $C_{k+1}$ 之间的空隙的厚度的命题. 既然 $C_K \sub C_{k+1} ^ {\circ}$ 那么它们之间的距离存在一定厚度)\\
    因而 $B_i$ 和 $D_{i-2}$ 之间有一定距离.\
    对于 $B_i$ 中的每个 $x$ 我们都可以选取一个 centered at $x$ 的 closed cube, 并且使它包含在 $\sA$ 的某个元素中, 且 disjoint from $D_{i-2}$.\\
    通过取 finite subcover, 我们于是把 $B_i$ 用 finite 个 closed cubes contained in $A$ 给 cover 上了.\\
    这是 natural 的使用 ctbl 个 closed box(cube this time) 来 得到 open set 的方法.\\
    并且, 这个 \textbf{disjointness of each $B_i$ and $D_{i-2}$ gives local finiteness:} 对于任意一个 $x\in A$, $x \in D_i ^{\circ}$ for some $D_i$. 根据我们的构造: $D_i$ 至多只会 intersec cubes in $C_1, \cdots, C_{i+1}$. 
\end{proof}


\subsection{Partition of Unity on topological space}





\subsection{Partition of Unity Thm on $\bR^n$}



partition of unity 的意义:
这些 $\phi_i(x)$ 就像是权重函数. 对于任意一点,它们的和都是 1;并且,在一点附近只有有限个权重函数不是0,使得它具有计算意义。

Idea:我们想要把 open set 上的 real-valued 函数拆分成一个一个 box 上的,这样它的性质就变得非常好,以至于可以使用 Riemman Integration 中的所有工具,比如 Fubini's Thm.


我们之前的尝试:拆分这个 open set!
1. 把 open set 看作一个嵌套的 cpt set seq 的极限
问题:仍然不是 boxes. 并且没有拆分

2. 拆分成 almost disjoint closed boxes
问题: 首先, 没有 control over 每个 closed box 的 boundary, 不能把函数等价成这些可数个分段上的函数; 并且无限个分段的函数也不好处理.


我们发现只拆分定义域, 拆分成 almost disjoint closed boxes 已经是最好的情况了. 但是仍然不能很好地拆分一个函数.


而 Partition of Unity 则是一个完美的拆分函数的工具. 它拆分函数本身, 不是仅仅拆分定义域. 它并不严格地要求区间之间的分离程度, 而是通过构造和为 constant 1 的权重函数, 每一个权重函数都只在一小块区域上非0, 于是把函数转换成所有权重函数与自身乘积的累和; 并且, 我们要求这些权重函数的 supports 是 locally finite 的, 于是对于任意一点, 我们都能拆分函数成有限个权重函数与自身乘积的和. 

这样, 区间之间只需要 locally finite 就可以, 比 disjoint 宽松许多, 但是却在每点上都能写出函数的表达式. 


在 $\bR^n$ 上, 我们发现不仅可以构造 POU, 并且可以把性质做得更好, 做出 compact J-meas support (actually cube), 并且 $C^\infty$ 的 POU. 并且, 对于任意指定的 open covering, 我们都可以做出由它 dominate 的 POU.

这是有用的. 尤其是在积分中, 我们把一个开集上的函数上的积分, 拆分成 a seq of cubes 上的函数的积分和; 权重函数的 $C^\infty$ 性质使得它完全不影响原本函数的光滑性.




\subsection{Integration via POU}





\section{proof of COV}






\end{document}