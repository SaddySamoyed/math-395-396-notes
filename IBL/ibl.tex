\documentclass[lang=cn,11pt]{template}
\usepackage{amsmath}%
\usepackage{amssymb}%
\usepackage[UTF8]{ctex}
\title{395 IBL}

\begin{document}
\frontmatter
\tableofcontents
\mainmatter
%%%%%%%%%%%%%%%%%%%%%%%%%%%%%%%
%%%%%%%%%%%%%%%%%%%%%%%%%%%%%%%

\chapter{Baire Category Theorem}
%%%%%%%%%%%%%%%%%%%%%%%%%%%%%%%
%%%%%%%%%%%%%%%%%%%%%%%%%%%%%%%

\section*{1A: 证明 metric space 是 topological space}
\bold{Problem A:} Let $(X,d)$ be a metric space. Recall that a set $U$ is said to be open if for all $x\in U$ there exists $\epsilon>0$ such that $B_\epsilon(x) \subset U$.
\begin{enumerate}
\item Show that the open sets according to this definition define a topology on $X$. 
\item Show that a sequence $(x_n)$ in $X$ converges to $x_\infty$ in  this topology if and only if for all $\epsilon>0$ there exists an $N>0$ such that if $n\geq N$ then 
$$d(x_n, x_\infty)<\epsilon.$$
\item Show that  for all $x\in X$ and all $r>0$, the ``open ball"
$$B_r(x) = \{ y \in X : d(x,y)<r\}$$
is in fact an open set by the definition above. 
\item Show that  for all $x\in X$ and all $r>0$, the ``closed ball"
$$ \{ y \in X : d(x,y)\leq r\}$$
is in fact a closed set by the definition above. 
\item Give an example of a metric space $(X,d)$, a point $x$, and an $r>0$ such that 
$$\{ y \in X : d(x,y)\leq r\}$$
is not the closure of $B_r(x)$.

\begin{solution}
    考虑 discrete toplogy
\end{solution}
\end{enumerate}

Recall that a metric space is called complete if every Cauchy sequence converges. 

The Baire Category Theorem states the following: 

\begin{theorem}
    Let $(X,d)$ be a complete metric space, and let $(U_n)_{n=1}^\infty$ be a sequence of open dense sets in $X$. Then 
$$\bigcap_{n=1}^\infty U_n$$
is dense in $X$. 
\end{theorem}


\section*{1B: Baire Category Thm 在不 complete MS 中的反例}
Show that this is false without the assumption of completeness by considering the rationals with the usual metric. 
\begin{solution}
    考虑 
    $$
    {q_n} = \{ \text{all m/n with in lowest term } \}  $$
对于任意 $n \in \bN$
    $$
    U_n = \bQ \backslash \{q_n\}
    $$ 都是一个 dense and open set in $\bQ$. 但是
    $$
    \bigcap_ {i = 1} ^ \infty U_i = \emptyset
    $$
\end{solution}

\section*{1C: 证明 Baire Category Thm}

Prove the Baire Category Theorem as follows. 

\begin{enumerate} 
\item Show that it suffices to show that for all $x_0\in X$ and $r_0>0$ the intersection contains a point of the ball $B_{r_0}(x_0)$. 
\begin{proof}
    By def.
\end{proof}
\item With $r_i, x_i$ already having been defined, show that you can pick $x_{i+1} \in X, 0< r_{i+1}< r_i/2$ so that 
$$\overline{B_{r_{i+1}} (x_{i+1})} \subset B_{r_i}(x_i) \cap U_{i+1}.$$
\begin{proof}
    
\end{proof}
\item Show that the sequence $(x_i)$ is a Cauchy sequence. 
\item Show that the limit of $(x_i)$ is a point of $\bigcap_{n=1}^\infty U_n$ contained in $B_{r_0}(x_0)$.
\end{enumerate}

\section*{1D: complete perfect MS is unctbl}

Suppose that $(X,d)$ is a (non-empty) complete metric space in which every point is a limit point. Use the Baire Category Theorem to show $X$ is uncountable. 

\newpage
%%%%%%%%%%%%%%%%%%%%%%%%%%%%%%%
%%%%%%%%%%%%%%%%%%%%%%%%%%%%%%%
\chapter{Tempt to define measure}
%%%%%%%%%%%%%%%%%%%%%%%%%%%%%%%
%%%%%%%%%%%%%%%%%%%%%%%%%%%%%%%

\bold{Problem A:} Let us start with the interval $C=[0,1]$ and remove the middle third open interval $(\frac 13, \frac23)$. This leaves us with the set 
$$C_1=\left[0, \frac13\right]\cup \left[\frac23, 1\right]$$ formed of $2$ closed subintervals. Having constructed $$C_1\supset C_2\supset \cdots \supset C_n$$ where $C_n$ is the union of $2^n$ subintervals each of length $\frac{1}{3^n}$, we construct $C_{n+1}$ as follows: To obtain $C_{n+1}$ we remove the middle third of each of the $2^n$ intervals that form $C_n$. This leaves us with a union of $2^{n+1}$ intervals each of length $\frac{1}{3^{n+1}}$. Let $$C=\bigcap_{n=1}^{\infty} C_n.$$ This is called the (middle thirds) Cantor set. 

\begin{enumerate}

\item Show that $C$ is non-empty and compact.

\item Show that every point in $C$ is a limit point.  

\item Conclude using the last worksheet that $C$ is uncountable.

\item  Show that $C$ cannot contain any interval $(a, b)$. 

\item What is the total length of $C_n$? What would be a reasonable definition of the length of $C$?

\end{enumerate}

\begin{remark}
     Cantor set 是一个 compact 且 closed 的集合(甚至 perfect )(并且由于它在  $\mathbb{R}^n$ 中,\textbf{这表示它甚至是一个 complete metric space});并且它 uncountable;然而它却不包含任何开区间(我们知道它的 \textbf{Lebesgue measure = 0}).
    
\end{remark}

\bold{Problem B:} Motivated by the above, it would be grand to have a measure function that tells us how big or small a subset of $\R^d$ is. This would be a function from the set $\mathcal P(\R^d)$ of subsets of $\R^d$ into $[0, \infty]$, say $$m: \mathcal P(\R^d)\to [0,\infty].$$ We would like this function to satisfy the following properties: 

\begin{enumerate}
\item[a)] If $E_1, E_2, \ldots$ is a countable collection of disjoint subsets of $\R$, then 
$$
m(\cup_{n=1}^\infty E_n)=\sum_{n=1}^\infty m(E_n).
$$
\item[b)] If $E$ is congruent to $F$ (i.e. $F$ can be obtained from $E$ by applying rigid motions: translations, rotations, or reflections) then we should have that $m(E)=m(F)$.
\item[c)] $m([0,1)^d)=1$.
\end{enumerate} 

The bad news is that no such function can exist, and here's why (at least when $d=1$). Let us define an equivalence relation between elements of $[0,1)$ as follows: We say $x\sim y$ if $x-y$ is a rational number. Let $N$ be a subset of $[0,1)$ that contains exactly one element of each equivalence class (the existence of such a $N$ requires invoking the axiom of choice). Now let $$R=[0,1)\cap \Q,$$ and for each $r\in R$ define the set 
$$
N_r=\{x+r: x\in N\cap [0,1-r)\}\cup \{x+r-1: x\in N\cap [1-r, 1)\}.
$$
(Basically $N_r$ is just the translate of $N$ by $r$ units to the right, except that we move the part that sticks out of the interval $[0,1)$ one unit to the left).
\begin{enumerate}
\item Show that $[0,1)$ is the disjoint union of $N_r$ for $r\in R$.
\item Show that if a measure function satisfying a), b) and c) above exists, then $m(N)=m(N_r)$ for every $r\in R$.
\item Arrive at a contradiction.
\end{enumerate}
\begin{proof}
\noindent 我们想要测量一个 $\mathbb{R}^n$ 子集的“长度”。我们想要这个测量函数满足:

\begin{enumerate}
    \item  可数个集合的 measure closed under addition;
    \item  Congruent (通过 translate 或者旋转或者反射) 的集合的 measure 相同; 
    \item  精准地测量出 $m([0,1)) = 1$. 
\end{enumerate}


\noindent 但是实际上我们发现不可能存在这样的函数. 
\noindent 归谬过程:
   \begin{enumerate}
       \item   我们把 $[0,1)$ 之间所有可以通过加减有理数得到的元素都归为一个 congurent class (很直观地说,所有的有理数都会进入一个 congruent class,而最简根数不同的无理数以及各种 transcendental numbers 会形成各自的 congruent classes.)
       \item  把 $[0,1)$ 之间的所有有理数放进一个集合 $R$,在每个 congruent class 中取一个元素,放进一个集合 $N$,我们会发现:任取 $r \in R$,通过把 $N$ 整体地 translate $r$ 个大小并把放不下的放到左边(也就是一个 circular translate),我们得到 $N_r$,会发现 $\bigsqcup_{r} N_r = [0,1)$;并且每个 $N_r$ 的 measure 都是一样的.
       \item   如果 $m(N)$ 不等于 0,那么 $m([0,1)) = \infty$;如果 $m(N)$ = 0,那么 $m([0,1)) = 0$;
   \end{enumerate}
\end{proof}
\bold{Remark:} One might think that possibly relaxing condition a) to cover only \emph{finitely} many disjoint sets $E_n$, i.e. 
        $$
m(\cup_{n=1}^N E_n)=\sum_{n=1}^N m(E_n).
$$    
would resolve the contradiction. Unfortunately, the Banach-Tarski paradox tells us that this is not enough to resolve this issue. 

Banach-Tarski tells us that we can split the unit ball in $\R^3$ into finitely many (actually 5 is sufficient) many disjoint pieces, apply rigid motions to those pieces and then reassemble them to obtain two copies of the unit ball.

\bold{Conclusion:} The problem with the above wishlist is that we insisted on being able to measure \emph{every} subset of $\R^d$. We have shown that this is impossible. The solution is to be content with a measure function that is defined on some but not all subsets. Such subsets will be called measurable subsets.



\newpage
%%%%%%%%%%%%%%%%%%%%%%%%%%%%%%%
%%%%%%%%%%%%%%%%%%%%%%%%%%%%%%%
\chapter{Pixel measure}
%%%%%%%%%%%%%%%%%%%%%%%%%%%%%%%
%%%%%%%%%%%%%%%%%%%%%%%%%%%%%%%

\bold{Definitions:} An interval $I$ is a subset of $\R$ of the form $[a,b], [a,b), (a, b]$, or $(a,b)$ where $a, b \in \R$. The length of $I$ is defined to be $|I|:=b-a$. 

A \emph{box} in $\R^d$ is a Cartesian product of intervals $$B=I_1 \times \cdots \times I_d$$ and its \emph{volume} is defined to be $$|B|=|I_1|\times \cdots \times |I_d|.$$ 

An \emph{elementary set} is any subset of $\R^d$ which is the union of a finite number of boxes. 

\bold{Problem A:} Show that if $E, F\subset \R^d$ are elementary sets, then the union $E\cup F$, the intersection $E\cap F$, the set theoretic difference $E\setminus F$, and the symmetric difference $E\Delta F=(E\setminus F)\cup (F\setminus E)$ are also elementary. Also, if $x\in \R^d$, then the translate $E+x:=\{y+x: y \in E\}$ is also elementary. 

\bold{Problem B:} Show that any elementary set $E$ can be expressed as the finite union of disjoint boxes. (Hint: Start with $d=1$.)


\bold{Definition:} Let $E$ be an elementary set. The above question allows to write $$E=B_1 \cup B_2 \cup \cdots \cup B_n,$$ where $B_1, \ldots, B_n$ are disjoint. We define the elementary measure of $E$ as $$m(E):=|B_1|+|B_2|+\cdots+|B_n|.$$

\bold{Problem C:} Show that $m(E)$ is well-defined in the sense that if $E$ can be expressed in two ways as a union of disjoint boxes $B_1, \ldots, B_n$ and $B_1', \ldots, B_m'$, then 
 
 $$|B_1|+|B_2|+\cdots+|B_n|=|B_1'|+|B_2'|+\cdots+|B_m'|.$$

There's more than one approach you can take, but here you should use the following approach: First prove that for an interval $I$ in $\R$,  
$$
|I|=\lim_{N\to \infty} \frac{1}{N}\# \left(I\cap \frac{1}{N} \Z\right),
$$
where $\#S$ denotes the number of elements of a set $S$. 
And more generally for a box $B$, prove that
$$
|B|=\lim_{N\to \infty} \frac{1}{N^d}\# \left(B\cap \frac{1}{N} \Z^d\right).
$$
Here $\frac{1}{N} \Z^d=\{\frac{k}{N}: k \in \Z^d\}$. Use this to give an alternative definition of $m(E)$ for an elementary set that does not rely on its decomposition into disjoint boxes.

\bold{Problem D:} Show that the following holds
\begin{enumerate}
\item If $E_1, \ldots, E_n$ are disjoint elementary sets, then 
$$
m(E_1 \cup \cdots \cup E_n)=\sum_{i=1}^n m(E_i)
$$
Recall that this is called finite additivity.

\item If $E\subset F$ are two elementary sets, then 
$$
m(E) \leq m(F).
$$ 
This property is called monotonicity.

\item Show that if $E_1, E_2, \ldots, E_n$ is an arbitrary finite collection of elementary sets, then 
$$
m(E_1\cup \cdots \cup E_n)\leq m(E_1) +\cdots+ m(E_n).
$$
This is called finite subadditivity.
 \end{enumerate}
 
\bold{Why is this unsatisfactory?} Of course, the main problem with this measure is that we can only measure relatively simple sets (namely the elementary sets). For example, we cannot measure the area of a disc. 

\bold{Definition:} One might be tempted to generalize this measure naively as follows: For an arbitrary set $E \subset \R^d$, define 
 $$
 m_{\mathrm{pixel}}(E)=\lim_{N\to \infty} \frac{1}{N^d}\# \left(E\cap \frac{1}{N} \Z^d\right).
 $$
 
This is not a particurlary satisfactory definition either. One reason (out of many) is the following: 
 
 \bold{Problem E:}
Find a subset $E$ of $\R$ such that both $m_{\mathrm{pixel}}(E)$ and $m_{\mathrm{pixel}}(E+x)$ exist, but $m_{\mathrm{pixel}}(E)\neq m_{\mathrm{pixel}}(E+x)$ for some $x\in \R$.



\newpage
%%%%%%%%%%%%%%%%%%%%%%%%%%%%%%%
%%%%%%%%%%%%%%%%%%%%%%%%%%%%%%%
\chapter{Jordan measure}
%%%%%%%%%%%%%%%%%%%%%%%%%%%%%%%
%%%%%%%%%%%%%%%%%%%%%%%%%%%%%%%


The main caveat of elementary measure is that it only allows us to measure elementary sets, which is a fairly restrictive family of sets. Building on the old intuition (going back at least to Archimedes) we can lower bound (respectively upper bound) the measure of a set by approximating it from within (respectively without) by an elementary set, i.e. if $A$ and $B$ are elementary and $A\subset E \subset B$, then the measure of $E$ (if it exists) should be sandwiched between that of $A$ and $B$. 

\bold{Definitions:} Let $E \subset \R^d$ be a bounded set. 
 \begin{itemize}
 \item The \emph{Jordan inner measure} $\underline m_J(E)$ of $E$ is defined as 
 $$
 \underline m_J(E)=\sup_{A\subset E, A\,  \textrm{elementary}} m(A).
 $$
 Here $m(A)$ is the elementary measure of $A$.
 
 \item The \emph{Jordan outer measure} $\overline m_J(E)$ of $E$ is defined as 
 $$
 \overline m_J(E)=\inf_{A\supset E, A\,  \textrm{elementary}} m(A).
 $$
\item If $\underline m_J(E)=\overline m_J(E)$, we say that $E$ is Jordan measurable, and call the common value $m(E)$ (the Jordan measure of $E$).

By convention, we do not consider unbounded sets to be Jordan measurable.  
\end{itemize}


\bold{Problem A:} Assume that $E \subset \R^d$ is bounded. Show that the following are equivalent:
 \begin{enumerate}
 \item[a)] $E$ is Jordan measurable. 
 \item[b)] For every $\epsilon>0$, there exists elementary sets $A$ and $B$ such that $A\subset E \subset B$ and $m(B\setminus A)\leq \epsilon$.
 \item[c)] For every $\epsilon>0$, there exists an elementary set $A$ such that $\overline m_J(E\Delta A)\leq \epsilon$.
 \end{enumerate}

\bold{Problem B:} Deduce that every elementary set $E$ is Jordan measurable and that its Jordan measure is the same as its elementary measure. In particular, $m(\emptyset)=0$.


\bold{Problem C:} Let $E, F$ be Jordan measurable sets.  Clearly $m(E) \geq 0$. Show that 
\begin{enumerate}
\item  $E\cup F, E\cap F, E\setminus F, $ and $E\Delta F$ are all Jordan measurable. 
\item (Finite additivity) If $E$ and $F$ are disjoint, then $m(E\cup F)=m(E)+m(F)$.
\item (Monotonicity) If $E \subset F$, then $m(E)\leq m(F)$. 
\item (Finite subadditivity) $m(E\cup F) \leq m(E)+m(F)$. 
\item (Translation invariance) for any $x\in \R^d$, $m(E+x)=m(E)$.
 \end{enumerate}

 
 
\bold{Problem D:} Let $B$ be a closed box of $\R^d$ and $f: B\to \R$ a continuous function. 

 \begin{enumerate}
 \item Show that the graph $\{(x, f(x)): x\in B\}\subset \R^{d+1}$ is Jordan measurable in $\R^{d+1}$ and that it has Jordan measure 0. \emph{Hint: Use that $f$ is uniformly continuous.}
 \item Show that the set $\{(x, t):x\in B, 0\leq t \leq f(x)\}\subset \R^{d+1}$ is Jordan measurable. 
\end{enumerate} 

From this we conclude that some familiar sets like triangles in $\R^2$ and balls in $\R^d$ are Jordan measurable. 

\bold{Problem E:} Is $\bQ \cap [0,1]$ Jordan measurable? What are its inner and outer Jordan measure?




\newpage
%%%%%%%%%%%%%%%%%%%%%%%%%%%%%%%
%%%%%%%%%%%%%%%%%%%%%%%%%%%%%%%
\chapter{Jordan measurability and Riemann Integrability}
%%%%%%%%%%%%%%%%%%%%%%%%%%%%%%%
%%%%%%%%%%%%%%%%%%%%%%%%%%%%%%%


\bold{Problem A:} Show that the open and closed balls of radius $r$ in $\bR^d$ are both Jordan measurable, and that their Jordan measure is $c_d r^d$ for some constant $c_d>0$ that only depends on the dimension.
 
%\emph{Hint: A half ball can be written as the region below a graph} %Another Show that the ball is the finite union of (possibly not disjoint) subsets of the form described in {\bf Q2)} above. You can use without proof that if $E$ is Jordan measurable and $L$ is a rotation then $L(E)$ is also Jordan measurable.}  

\bold{Problem B (optional):} Establish the bound $\left(\frac{2}{\sqrt d}\right)^d \leq c_d\leq 2^d$. 

\bold{Problem C:} Let $E \subset \R^d$ be bounded. Show that both $E$ and its closure $\overline E$ have the same Jordan outer measure. 

\bold{Problem D:} Let $E \subset \R^d$ be bounded.  Show that $E$ and its interior have the same Jordan inner measure. 

\bold{Problem E:} Show that $E$ is Jordan measurable if and only if the topological boundary $\partial E=\overline E\setminus E^\circ$ has Jordan outer measure 0. Here $E^\circ$ denotes the interior of $E$. 

\bold{Recall:} To define the Riemann integral of a {bounded} function $f$ on an interval $[a,b]\subset \R$, we first recall the notion of a partition $\mathcal P$, which is a set of points $$x_0=a<x_1<x_2<\cdots <x_n=b.$$ The norm of the partition is $\Delta \mathcal P=\max_{1\leq k \leq n} x_k-x_{k-1}$, and we denote by $\Delta x_k=x_k-x_{k-1}$. For each such partition, we define to quantities:
 $$
 L(f, \mathcal P)=\sum_{k=1}^n  \Delta x_k \inf_{[x_{k-1}, x_k]} f, \quad \text{and} \quad U(f, \mathcal P)=\sum_{k=1}^n \Delta x_k \sup_{[x_{k-1}, x_k]} f.
 $$
We define the lower and upper Darboux integrals respectively as
 $$
 \underline{\int_a^b} f(x) dx=\sup_{\mathcal P} L(f, \mathcal P), \quad and \quad \overline{\int_a^b} f(x) dx=\inf_{\mathcal P} U(f, \mathcal P).
 $$
 where the extrema above are taken over all partitions of the interval $[a,b]$. We say that $f$ is Riemann integrable if the above two numbers are equal. We define the common value as the Riemann (or Darboux) integral of $f$.
 
\bold{Problem F:} Let $[a,b]$ be an interval and let $f:[a,b]\to \R$ be a bounded nonnegative function. Show that $f$ is Riemann integrable if and only if the set $E:=\{(x,t): x\in [a,b]: 0\leq t \leq f(x)\}$ is Jordan measurable in $\R^2$.



\newpage
%%%%%%%%%%%%%%%%%%%%%%%%%%%%%%%
%%%%%%%%%%%%%%%%%%%%%%%%%%%%%%%
\chapter{Lebesgue outer measure}
%%%%%%%%%%%%%%%%%%%%%%%%%%%%%%%
%%%%%%%%%%%%%%%%%%%%%%%%%%%%%%%


\bold{Where we are right now?} We have thus far discussed the classical theory of Jordan measure, which went as follows:
 \begin{enumerate}
 \item We define the notion of a box and its volume $|B|$ or $v(B)$,
 \item Then we defined the notion of an elementary set and its elementary measure,
 \item Then we defined the notion of Jordan inner and outer measure $\underline m_J(E)$ and $\overline m^J(E)$ and said that a set $E$ is Jordan measurable if those two concepts agree. 
 \end{enumerate}

 In particular, unwinding the definition of the Jordan outer measure, we have that for any set $E$
 $$
 \overline{m}_J(E)= \inf_{E\subset B_1 \cup \cdots \cup B_k} |B_1|+\cdots+|B_k|
 $$
 where the infimum is taken over all finite coverings of $E$ by boxes $B_1, \ldots, B_k$.

\section*{6A: Jordan measurable 的等价表述}
Show that a set $E$ is Jordan measurable if and only if for every $\epsilon>0$ there exists an elementary set $U$ containing $E$ such that $\overline{m}_J(U\setminus E)<\epsilon$.
\begin{proof}
    Follows from 5E.\\
    Jordan measurable 的意义是可以 well-approximate by elem set.
\end{proof}

\begin{definition}{Lebesgue outer measure}
The notions of Lebesgue outer measure and Lebesgue measurability are refinements of the Jordan ones as follows:

 We modify the notion of Jordan outer measure by replacing the finite union of boxes by a countable union of boxes, i.e.

$$
m^*(E)=\inf_{E\subset \cup_{j=1}^\infty B_j} \sum_{j=1}^\infty |B_j|
$$
where the union above is taken over boxes $B_j \subset \R^d$. This is the Lebesgue outer measure of $E$. 

\end{definition}


\section*{6B: $m^*(E) \leq \overline{m_J} (E)$}
Show that $m^*(E)\leq \overline{m}_J(E)$ where $\bar m_J$ is the Jordan outer measure. 

\begin{proof}
    extend finite to countable by continuing the seq using empty sets 即可得到.
\end{proof}


\section*{6C: Lebesgue outer measure 的定义可以 restrict to open/clsed boxes}
\bold{Problem C:} Show that in the definition above the countable cover by boxes in the definition of $m^*(E)$ can be restricted to closed boxes or open boxes.
\begin{proof}
    注意到 open box 和 closed box 的 boundary 的 Jordan measure 都是 0. 我们可以利用这个这个特点在 open box 内用大小差别小于 $\epsilon \over {2^n}$ 的 closed box 进行覆盖,最后得到任意开覆盖的测度和都 $\geq$ 闭覆盖的测度和的 inf; dually 也可得. 
\end{proof}


\section*{6D: countable set 的 Lebesgue outer measure 总是 0}
Show that the Lebesgue outer measure $m^*(E)$ is zero for any countable set $E$. Contrast this to fact that the Jordan outer measure of the rationals in $[0,1]$ was equal to 1.
\begin{proof}
    Trivial. 对于每一个点进行 $ < \epsilon \over {2^n}$ 的 box 覆盖即可. \\
    这说明 Lebesgue measure 是比 Jordan measure 更加好的.
\end{proof}


\begin{definition}{Lebesgue measurability}
A set $E\subset \R^d$ is said to be Lebesgue measurable if for every $\epsilon>0$, there exists an open set $U\subset \R^d$ containing $E$ such that $m^*(U\setminus E)\leq \epsilon$. If $E$ is measurable, we refer to $m(E)=m^*(E)$ as the Lebesgue measure of $E$. 
\end{definition}

\begin{remark}
    \begin{enumerate} 
\item Note that there is no need for $E$ to be bounded for this definition to make sense. 
\item The notion of Lebesgue measurability can be seen as a (finite to countably infinite) generalization of that of Jordan measurability since it can be shown that every open set is the countable union of closed boxes.  
\end{enumerate}
\end{remark}


\section*{6E: Monotonicity and Countable subadditivity}
1. Show that $m^*(\emptyset)=0$.
\begin{proof}
    trivial
\end{proof}
2. (Monotonicity) Show that if $E\subset F\subset \R^d$, then $m^*(E) \leq m^*(F)$.
\begin{proof}
    trivial. 每个 F 的覆盖也覆盖了 E.
\end{proof}
3. (Countable subadditivity) If $E_1, E_2, \ldots \subset \R^d$ is a countable sequence of sets, then $m^*\left(\cup_{n=1}^\infty E_n\right)\leq \sum_{n=1}^\infty m^*(E_n)$. 

\begin{proof}
    我们可以给序列中每个集合都创造一个可数覆盖. 这样,我们就会创造一个 
    $$
    \bigcup_{n=1}^{\infty} E_n \subseteq \bigcup_{n=1}^{\infty} \bigcup_{k=1}^{\infty} B_{n,k}
    $$
    我们可以用 $\epsilon \over {2^n}$ 来 bound 每个集合的覆盖和与它的 Lebesgue 外测度的距离,这样就可以把双累加变成单累加.
\end{proof}


\newpage
%%%%%%%%%%%%%%%%%%%%%%%%%%%%%%%
%%%%%%%%%%%%%%%%%%%%%%%%%%%%%%%
\chapter{Basic Properties of Lebesgue outer measure}
%%%%%%%%%%%%%%%%%%%%%%%%%%%%%%%
%%%%%%%%%%%%%%%%%%%%%%%%%%%%%%%


\bold{Recall:}  The Lebesgue outer measure of $E\subset \R^d$ is 
$$m^*(E)=\inf_{E\subset \cup_{j=1}^\infty B_j} \sum_{j=1}^\infty |B_j|
$$
where the union above is taken over boxes $B_j \subset \R^d$. 

 A set $E\subset \R^d$ is said to be Lebesgue measurable if for every $\epsilon>0$, there exists an open set $U\subset \R^d$ containing $E$ such that $m^*(U\setminus E)\leq \epsilon$. If $E$ is measurable, we refer to $m(E)=m^*(E)$ as the Lebesgue measure of $E$. 

Last time you proved monotonicity and countable sub-additivity.  

\section*{7A: dist>0 的集合外测度 union additive}
Show that if $\mathrm{dist}(E,F)>0$, then $$m^*(E\cup F)=m^*(E)+m^*(F).$$
 



\section*{7B: elementary set 的 Lebesgue measure 就是 elementary measure}
Show that if $E$ is an elementary set, then $m^*(E)=m(E)$ where $m(E)$ is the elementary measure of $E$ defined before.
\begin{proof}
    $m^*(E) \leq m(E)$ 显然. 
    \\要证明 $m(E) \leq m^*(E)$, 我们对任意 ctbl covering 取一个 disjoint cover   
\end{proof}





\section*{7C: Lebesgue measure 的大小处于 Jordan outer/inner measure 之间}
Conclude that if $E$ is any bounded set, then $$\underline m(E) \leq m^*(E) \leq \overline m (E)$$ where $\underline m(E)$ and $\overline m(E)$ are the inner and outer Jordan measures of $E$. 



\section*{7D: non Jordan measurable 的 open set}
Construct a bounded open subset $U$ of $\R$ that is not Jordan measurable. \emph{Hint: Start with an enumeration of the rationals in $[0,1]$ and create an open set whose Lebesgue outer-measure is arbitrarily small but the Jordan outer measure is $\geq 1$.}

\begin{solution}
    我们首先 list 出 [0,1] 之间的 ratioals, 称为 $(q_n)$. 我们对于每个 
\end{solution}


\section*{7E: ctbl 个 almost disjoint boxes 外测度 union additive}
\bold{Problem E:} We say that two boxes are \emph{almost disjoint} if their interiors are disjoint. Let $E=\cup_{n=1}^\infty B_n$ be a countable union of almost disjoint boxes $B_k$. Show that 
$$
m^*(E)=\sum_{k=1}^\infty |B_k|.
$$
\begin{proof}
By the monotinicity of Legesgue outer measure, it suffices to prove: the Lebesgue outer measure of unions of $B_k$ is greater than or equal to the sum of volumes of $B_k$. And, since volume does not consider openness/closeness, it suffices to prove that the Lebesgue outer measure of unions of $B_k$ is greater than or equal to the sum of volumes of interiors of $B_k$
\end{proof}



\newpage
%%%%%%%%%%%%%%%%%%%%%%%%%%%%%%%
%%%%%%%%%%%%%%%%%%%%%%%%%%%%%%%
\chapter{outer regularity}
Recall:
The Lebesgue outer measure of $E \subset \mathbb{R}^d$ is

\[
m^*(E) = \inf_{E \subset \bigcup_{j=1}^{\infty} B_j} \sum_{j=1}^{\infty} |B_j|
\]

where the union above is taken over boxes $B_j \subset \mathbb{R}^d$. A set $E \subset \mathbb{R}^d$ is said to be Lebesgue measurable if for every $\varepsilon > 0$, there exists an open set $U \subset \mathbb{R}^d$ containing $E$ such that 

\[
m^*(U \setminus E) \leq \varepsilon.
\]

If $E$ is measurable, we refer to $m(E) = m^*(E)$ as the Lebesgue measure of $E$.

So far, you've proved monotonicity and countable sub-additivity; the Lebesgue measure agrees with elementary measure on elementary sets; and that $m^*$ lies between Jordan inner and outer measure. You also understood Lebesgue measure for countable unions of almost disjoint boxes, where the Lebesgue measure is the sum of the volumes of the boxes.

\section*{8A: $\bR^n$ 中任意开集都是一个 ctbl union of almost disjoint boxes}

Show that any open subset $E$ of $\mathbb{R}^d$ can be written as the countable union of almost disjoint boxes (even countable union of almost disjoint closed cubes).


\begin{proof}
实际上与 $\bR$ 上的用 ctbl closed intervals 来逼近任意一个 open interval 如出一辙. 下面的 process 给出了一个更加 generalized 的算法.\\
Let $G \subset \mathbb{R}^n$ be open. \\
We construct a family of boxes as follows: 

Let
$$
R_1 = \{ [a_1 - 1, a_1] \times \cdots \times [a_d -1, a_d] \bigg| a_1, \cdots, a_d \in \mathbb{Z} \}
$$
$$
Q_1 = \{B \in R_1 \bigg| B \subset O\}
$$
And for each $n \geq 2$, let
$$
R_n = \{  [\frac{a_1 - 1}{2^n}, \frac{a_1}{2^n}] \times \cdots \times  [\frac{a_d-1}{2^n}, \frac{a_d-1}{2^n}] \bigg| a_1, \cdots, a_d \in \mathbb{Z}\}
$$
and $$
Q_n = \{ B \in R_n \bigg| B \subset O \backslash (\bigunion_{N=1}^{n-1} Q_N) \}
$$
Previously we have proved that $\bigunion_{n \in \mathbb{N}} Q_n = O$.\\
\end{proof}


\section*{8B: outer regularity} 
Let $E \subset \mathbb{R}^d$ be an arbitrary set. Show that

\[
m^*(E) = \inf_{E \subset U, \, U \text{ open}} m^*(U).
\]

This is called outer regularity.

\section*{8C: outer regularity 的 dual 并不正确} 
Give an example of a set $E \subset \mathbb{R}^d$ such that the reverse statement

\[
m^*(E) = \sup_{U \subset E, \, U \text{ open}} m^*(U)
\]

is false. (We will see that the right version of inner regularity is obtained by approximating the set $E$ by compact sets contained in it.)

%%%%%%%%%%%%%%%%%%%%%%%%%%%%%%%
%%%%%%%%%%%%%%%%%%%%%%%%%%%%%%%

\newpage
%%%%%%%%%%%%%%%%%%%%%%%%%%%%%%%
%%%%%%%%%%%%%%%%%%%%%%%%%%%%%%%
\chapter{Lebesgue measurability}

Recall: The notion of Lebesgue outer measure of a set \( E \):

\[
m^*(E) = \inf_{E \subset \cup_{j=1}^{\infty} B_j} \sum_{j=1}^{\infty} |B_j|
\]

where the union above is taken over boxes \( B_j \subset \mathbb{R}^d \). A set \( E \subset \mathbb{R}^d \) is said to be Lebesgue measurable if for every \( \epsilon > 0 \), there exists an open set \( U \subset \mathbb{R}^d \) containing \( E \) such that \( m^*(U \setminus E) \leq \epsilon \). If \( E \) is measurable, we refer to \( m(E) = m^*(E) \) as the Lebesgue measure of \( E \).

We have proven the following facts:

\begin{enumerate}
    \item Properties of the outer measure
    \begin{itemize}
        \item \( m^*(\emptyset) = 0 \).
        \item (Monotonicity) If \( E \subset F \subset \mathbb{R}^d \), then \( m^*(E) \leq m^*(F) \).
        \item (Countable subadditivity) If \( E_1, E_2, \ldots \subset \mathbb{R}^d \) is a countable sequence of sets, then
        \[
        m^*\left(\cup_{n=1}^{\infty} E_n\right) \leq \sum_{n=1}^{\infty} m^*(E_n).
        \]
    \end{itemize}
    \item If \( \text{dist}(E, F) > 0 \), then \( m^*(E \cup F) = m^*(E) + m^*(F) \).
    \item If \( E \) is an elementary set, then \( m^*(E) = m(E) \) where \( m(E) \) is the elementary measure of \( E \) defined before.
    \item Let \( E = \cup_{n=1}^{\infty} B_n \) be a countable union of almost disjoint boxes \( B_k \) (this means that their interiors are disjoint), then
    \[
    m^*(E) = \sum_{k=1}^{\infty} |B_k|.
    \]
    As such, \( \mathbb{R}^d \), for example, has infinite outer measure.
    \item Let \( E \subset \mathbb{R}^d \) be an arbitrary set. Then
    \[
    m^*(E) = \inf_{E \subset U, U \text{ open}} m^*(U).
    \]
    This is called outer regularity.
\end{enumerate}

\section*{9A: Null Sets are Measurable}
\begin{theorem}
Every set of Lebesgue outer-measure 0 is measurable.(such sets are called null sets)
\end{theorem}
\begin{proof}
    trivial.
\end{proof}

\section*{9B: Union of Lebesgue Measurable Sets is Lebesgue measurable} 
\begin{theorem}
If \( E_1, E_2, E_3, \ldots \subset \mathbb{R}^d \) are a sequence of Lebesgue measurable sets, then their union \( \cup_{n=1}^{\infty} E_n \) is also Lebesgue measurable.
\end{theorem}
\begin{proof}
    $\bN^2$ 也是 ctbl 的.\\
    我们对每个 $E_n$ 都选取一个 Open cover, 最后 double union 也是个 countable Open cover.\\
    选取任意 $\epsilon$
    由于每个 $E_n$ 都是 L-measurable 的,我们用 $\epsilon \over {2^n}$ 来 bound 每个 $E_n$ 和它的 cover 的差距. 即可得证.
\end{proof}



Try finish the proof the my two claims, and then finish the proof of the whole theorem
\section*{9C: closed set is Lebesgue measurable}
\begin{theorem}
Every closed set in \( \mathbb{R}^d \) is Lebesgue measurable.
\end{theorem}
Hint: Reduce to the compact case. Then, use that any open set is the countable union of almost disjoint closed cubes, as well as some of the properties reviewed above.
\begin{proof}
It suffices to prove that compact sets in $\mathbb{R}^d$ are Lebesgue-measurable, since even if $E$ is not bounded, we can write    $$E =  \bigunion_{n \in \mathbb{N}} (E \cap [-n, n]) $$
where each set is closed and bounded thus cpt. As long as we prove the statement for compact sets, we can get $E$ is Lebesgue measurable since countable union of Lebesgue measurable sets is Lebesgue measurable.\\
\textbf{Proof for compact sets:} Let $K \in \bR ^d $ be cpt, Then $O = K^c$ is open.\\

\end{proof}


\section*{9D: complement of Lebesgue measurable set is Lebesgue measurable} 
\begin{theorem}
If \( E \subset \mathbb{R}^d \) is Lebesgue measurable, then its complement \( \mathbb{R}^d \setminus E \) is also Lebesgue measurable.
\end{theorem}


\section*{9E: Intersection of Lebesgue measurable Sets is Lebesgue measurable}
\begin{theorem}
If \( E_1, E_2, E_3, \ldots \subset \mathbb{R}^d \) are a sequence of Lebesgue measurable sets, then their intersection \( \cap_{n=1}^{\infty} E_n \) is also Lebesgue measurable.
\end{theorem}




\chapter{Equiv Conditions for measurablity and Countable Additivity}
\begin{remark}
We now know open and closed sets are Lebesgue measurable, as are countable unions or intersections of Lebesgue measurable sets. 
We know finite additivity for sets that have positive distance to each other, and countable additivity for almost disjoint boxes.
\end{remark}



\section*{Lebesgue measurability: can be well approximated by an open/closed set}


\subsection*{10A: Symmetric difference}
If \( A, B, C \) are subsets of a set, show:
\[
A \Delta B \subset (A \Delta C) \cup (C \Delta B).
\]
\begin{remark}
def of symmetric diff: 
$$
A \Delta B =( A \cup B ) \backslash (A \cap B) = (A \backslash B) \cup (B\backslash A)
$$
sym diff 越加入更多 set 越大.
\end{remark}



\subsection*{10B: L-meas $\eq$ approximatable by open Sets}
\begin{theorem}
    A set \( E \) is measurable if and only if for every \( \epsilon > 0 \), one can find an open set \( U \) such that \( m^*(E \Delta U) \leq \epsilon \). In other words, \( E \) differs from an open set by a set of outer measure \( \epsilon \).
\end{theorem}


\begin{remark}
To show the difficult direction:\\
和 9D 一样:当我们想要两个相近似的集合最好有包含关系,但是根据已知的条件又构造不出包含关系的时候,我们可以通过近似条件构造一个 measure 无限接近的序列, 并且通过 intersection 来获得一个 \textbf{measure 0 set}, 最后通过交并补的方式来得到想要的近似且包含的关系.
\begin{enumerate}
    \item Fix \( \epsilon > 0 \). For each \( n \), let \( U_n \) be an open set with \( m^*(E \Delta U_n) \leq \frac{\epsilon}{2^{n+67}} \).
    \item \( U = \bigcup U_n \) is open. 且普通的 set diff 的 measure 总是小于等于 sym diff 的大小, we have \( m^*(U \backslash E) \leq ( m^*(U \Delta E)  \leq \sum_n m^*(U_n \Delta E) \leq \frac{\epsilon}{2} \).
    \item Show that \( E \setminus \cup_n U_n \) has measure zero, and hence there is an open set \( V \) containing \( E \setminus \cup_n U_n \) with \( m^*(V) \leq \frac{\epsilon}{2} \).
    \item Consider the set \( U \cup V \).
\end{enumerate}
\end{remark}

\subsection*{10C}
If \( E \) is measurable, then:
\[
m^*(E) = \sup_R m(E \cap B_R(0)).
\]

\begin{remark}
Hint: Use outer regularity and that every open set can be written as a countable union of almost disjoint boxes.
\end{remark}

\subsection*{10D}
If \( E \) is measurable, then:
\[
m^*(E) = \sup_K m^*(K),
\]
where \( K \) ranges over all closed subsets of \( E \).

\begin{remark}
Hint: Prove that for all \( \epsilon > 0 \), there is an open subset \( U \) containing \( E^c \) with \( m^*(U \setminus E^c) < \epsilon \). Then take \( K = U^c \).
\end{remark}

\section*{10E: Inner regularity}
\begin{theorem}
If \( E \) is measurable, then:
\[
m^*(E) = \sup_K m^*(K),
\]
where \( K \) ranges over all compact subsets of \( E \). (This is called inner regularity.)
\end{theorem}





\subsection*{10F: Measurable $\eq$ approximatable by closed Sets}
A set \( E \) is measurable if and only if for every \( \epsilon > 0 \), one can find a closed set \( F \) such that \( m^*(E \Delta F) \leq \epsilon \). In other words, \( E \) differs from a closed set by a set of outer measure \( \epsilon \).

\section*{10G: Countable Additivity}
\begin{theorem}
If \( E_1, E_2, \dots \subset \mathbb{R}^d \) is a countable sequence of disjoint Lebesgue measurable sets, then
\[
m\left( \bigcup_{n=1}^{\infty} E_n \right) = \sum_{n=1}^{\infty} m(E_n).
\]
This property is known as countable additivity.
\end{theorem}




\section*{10H: Why we need $\epsilon \over {2^n}$}
Let \( a_{n,m} = \frac{1}{nm} \), for \( n \) and \( m \) positive integers. Show that:
\[
\inf_n \sum_m a_{n,m} \neq \sum_n \inf_m a_{n,m}.
\]

%%%%%%%%%%%%%%%%%%%%%%%%%%%%%%%
%%%%%%%%%%%%%%%%%%%%%%%%%%%%%%%

\end{document}
